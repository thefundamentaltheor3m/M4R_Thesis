\section{The Cauchy-Goursat Theorem}

\begin{comment}
Maybe begin with an anecdote - no sooner had we entered Hales's office in Pittsburgh than he asked about how we plan to deform integration contours.

Have 3 subsections.
1. Informal maths
2. Discussion of existing formalisation of closed rectangular case, with emphasis on why we don't have it for other cases (cite Hales's formalisation of the Jordan Curve Theorem in HOL-Light, maybe try and explain why we don't have something similar in Lean)
3. Discussion of our approach to the indefinite case (informally and formally)
Also maybe find better words than closed/open? Because these words are NOT used here in a topological sense, but rather in a very visual sense ("are the two endpoints of the curve the same point or are they not? Does the curve even have two endpoints or does it just have one and then go off to i\infty in the other?")
\end{comment}

\subsection{The Cauchy-Goursat Theorem for Bounded Rectangular Contours}

\subsection{The Cauchy-Goursat Theorem for Unbounded Rectangular Contours}

\subsection{Scope for Further Development}

% Mention we might be able to formalise a more general version of Cauchy-Goursat, eg. for triangles, by approximating using rectangular contours. But mention that this is quite unnecessary for this project, as everything can be done with rectangular contours.