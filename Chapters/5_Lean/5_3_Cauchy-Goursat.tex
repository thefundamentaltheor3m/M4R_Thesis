\section{The Cauchy-Goursat Theorem}
\label{Ch5:Sec:Cauchy-Goursat}

\begin{comment}
Maybe begin with an anecdote - no sooner had we entered Hales's office in Pittsburgh than he asked about how we plan to deform integration contours.

Have 3 subsections.
1. Informal maths
2. Discussion of existing formalisation of closed rectangular case, with emphasis on why we don't have it for other cases (cite Hales's formalisation of the Jordan Curve Theorem in HOL-Light, maybe try and explain why we don't have something similar in Lean)
3. Discussion of our approach to the indefinite case (informally and formally)
Also maybe find better words than closed/open? Because these words are NOT used here in a topological sense, but rather in a very visual sense ("are the two endpoints of the curve the same point or are they not? Does the curve even have two endpoints or does it just have one and then go off to i\infty in the other?")
\end{comment}

There are some areas of mathematics that are notoriously difficult to formalise. Algebra, for example, tends to be easier to formalise than analysis. Within analysis, it tends to be particularly difficult to formalise geometric ideas. The Jordan Curve Theorem, for example, tends to be a particularly difficult theorem to formalise. It was not until 2007 that this theorem, proposed in the late 19th Century, was formalised by Tom Hales \cite{JordanCurve} in HOL Light, and to this day, no formalisation exists in Lean. The author had the privilege of meeting Hales in Pittsburgh, USA, in March 2025 to discuss the formalisation of $8$-dimensional sphere packing in Lean, and the very first question Hales asked was what the strategy was to overcome the challenges of not having a Lean formalisation of the Jordan Curve Theorem. It turns out that there is a workaround, which we explore in this section.

Before we discuss the workaround, we briefly discuss the statement of the \JCT\ and its relevance to this project. The \JCT\ essentially states that a simple closed curve $C \subset \R^2$, given as the image of a continuous injection from $\mathbb{S}^1$, divides $\R^2 \setminus C$ into a bounded region, known as the \textit{interior} of $C$, and an unbounded region, known as the \textit{exterior} of $C$. The relevance of the \JCT\ is that in Viazovska's construction---specifically, in the proof that $a$ and $b$ satisfy the double zero property---it becomes necessary to deform contours. While the contours in question are not closed, the proof that the deformation is possible (under a vanishing condition) follows from limiting applications of the Cauchy-Goursat Theorem, which states that integral of $\C \to \C$ function around a closed contour is zero if the function is holomorphic in the interior. The point is that the \JCT\ is necessary to define the interior, and without some version of the \JCT, it becomes challenging to state the all-important holomorphicity condition of the \CGT.

While it may not appear, at first, that there \textit{is} a workaround, it turns out that weaker versions of the \CGT, where the interior is more easily defined, are sufficient for our purposes. We outline the bounded and unbounded versions of these below.

% Are these section names too informal? :)

\subsection{Rectangles}

\subsection{Circles and Rectangles}