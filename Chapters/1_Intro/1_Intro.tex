\chapter{Introduction}
\label{Ch1:Chapter}
\thispagestyle{empty}

On 5 July, 2022, in Helsinki, Finland, the International Mathematical Union announced the names of the four mathematicians who were to be awarded the Fields Medal, the most coveted prize in the world of mathematics: Hugo Duminil-Copin, June Huh, James Maynard and Maryna Viazovska. Duminil-Copin, Huh and Maynard received this most prestigious honour for making several outstanding contributions to their specific fields of expertise---respectively, statistical physics, geometric combinatorics, and analytic number theory. Viazovska, on the other hand, was awarded the Fields Medal for a more focused line of research: the pursuit if the great mysteries of $\R^8$ and $\R^{24}$, chief amongst them the optimality of the $E_8$ and Leech lattice sphere packings in these spaces. Her solution in dimension $8$ is particularly revolutionary: it uses insights from Fourier analysis and the theory of modular forms to construct a special function---the so-called Magic Function---that, in combination with a previous result by Cohn and Elkies, proves that the $E_8$ lattice packing is the densest possible sphere packing in $\R^8$. Very shortly afterwards, Cohn, Kumar, Miller, Radchenko and Viazovska were able to use similar ideas to prove that the Leech lattice packing is the densest possible sphere packing in $\R^{24}$.

Before Viazovska's remarkable breakthrough, the optimal sphere packing density was only known in dimensions $1$, $2$ and $3$ \cite{CohnOnViazovskaICM}. Furthermore, Thomas Hales' solution in dimension $3$ \cite{HalesKeplerInformal} was lengthy and involved extensive computer-assisted calculations; in contrast, Viazovska's proof in dimension $8$ is elegant and concise. Even before Viazovska was awarded the Fields Medal, her work received wide acclaim from eminent mathematicians across the world: Peter Sarnak described it as ``stunningly simple, as all great things are,'' and Akshay Venkatesh remarked that her Magic Function is very likely ``part of some richer story'' that connects to other areas of mathematics and physics \cite{QuantaPiece}. Viazovska's work is a truly remarkable achievement in modern mathematics, with its elegance coming from the manner in which the many pieces of the puzzle fit perfectly together. One of the goals of this project is to offer a detailed exposition of one of those pieces: the construction of the so-called `Magic Function' in dimension $8$.

\section{The Sphere Packing Problem}

The Sphere Packing problem is a classical optimsation problem in mathematics. The problem can be formulated as follows.

\begin{boxproblem}[The Sphere Packing Problem in Dimensinon $n$]\label{Ch1:Prob:SpherePacking_n}
    Given some $n \in \N$, what is the densest possible non-overlapping arrangement of $n$-spheres of equal radius in $\R^n$?
\end{boxproblem}

Despite its rather straightforward formulation, \Cref{Ch1:Prob:SpherePacking_n} is notoriously difficult to solve. Indeed, one obvious question that arises when one looks at the problem statement is how one might define the concept of density. It turns out that the definition is slightly unwieldy, though introducing a periodicity assumption on the sphere packing whose density one wishes to find considerably simplifies this problem.

A key challenge in solving the sphere packing problem in dimension $n$ is not defining and understanding the concept of sphere packing density but the fact that proceeding inductively yields suboptimal results. In other words, knowing what the solution to the sphere packing problem is in dimension $n$ does not, in general, amount to knowing what it is in dimension $n + 1$~\cite{CohnOnViazovska}.

One exception to the above is the solutions in dimension $2$ and $3$. 
\section{The Formalisation Movement}

While Hales announced his intent to formally verify his proof of the Kepler Conjecture in 2003, it was not till 2006, after Hales's solution appeared in the \textit{Annals}, that a formal description of Hales's formalisation project was published. Of his motivations, Hales wrote:
\begin{quote}
    \textit{In response to the lingering doubt about the correctness of the proof, at the beginning of 2003, I launched the \emph{Flyspeck} project, whose aim is a complete formal verification of the Kepler Conjecture. In truth, my motivations for the project are far more complex than a simple hope of removing residual doubt from the minds of few referees. Indeed, I see formal methods as fundamental to the long-term growth of mathematics.}~\cite{FlyspeckAnnouncement}
\end{quote}
Formal theorem proving was not unheard of in 2006. Interactive theorem provers, such as Coq and PRL, have existed since the 1980s. However, it was still a relatively young field, and the amount of mathematics that had been formalised was limited. Hales's project was immensely ambitious, and the fact that it succeeded, despite taking over a decade, is impressive.

There is something prophetic about Hales's ``far more complex'' motivations for launching the Flyspeck project. The field of formal theorem proving has grown rapidly in the last decade, and interactive theorem provers like Lean are slowly making their way into mainstream mathematics. An excellent example of this is the formal verification of Gowers, Green, Manners and Tao's proof of Marton's Conjecture~\cite{PFROriginalPaper}, which was formally verified in Lean in just three weeks. In particular, their proof was formally verified \textit{before} their paper was submitted for publication. The paper is set to appear in the \textit{Annals}.\todo{Update with actual publication date if published before thesis deadline. Also update citation accordingly.}

There are many advantages of formal theorem proving. One advantage is the fact that formally proved theorems are verified by a proof assistant. When code written in proof assistants is compiled, if there are no errors, then the proof can be thought of as being `correct', in the sense of  being consistent with the axioms of the proof assistant.\todo{Finish}

% Continue...
\section{The Scope of this Project}

% Ask how to word this... is "I" ok??

In November 2023, I had the privilege of meeting Maryna Viazovska while pursuing an exchange programme at the Swiss Federal Institute of Technology, Lausanne, where she is based. We began discussing formalising her solution to the sphere packing problem in $8$ dimensions, and soon initiated a collaboration with Christopher Birkbeck, Seewoo Lee, and Gareth Ma, with invaluable assistance from Kevin Buzzard, Utensil Song, and Patrick Massot. On 31 May 2024, Viazovska formally announced at the ICMS workshop \textit{Formalisation of Mathematics: Workshop for Women and Mathematicians of Minority Gender} that we would be attempting to formalise her groundbreaking paper.

Viazovska's original paper~\cite{Viazovska8} is divided into five sections. The first section introduces sphere packings and develops basic theory; the second discusses the Cohn-Elkies linear programming bounds~\cite[Theorem 3.1]{CohnElkies}; the third offers some background on the theory of modular forms; the fourth constructs two radial, Schwartz Fourier eigenfunctions with double zeroes at almost all points on the $E_8$ lattice; and finally, the fifth uses these eigenfunctions to construct the ``Magic Function'', a Schwartz function that satisfies the conditions of Cohn and Elkies's theorem to give an upper bound for all sphere packings in $\R^8$ that is equal to the density of the $E_8$ packing. The first two sections were formalised collaboratively in July and August 2024, and the third section is actively being worked on by Birkbeck and Lee. This project focuses on formalising the fourth and fifth sections of Viazovska's paper. The code written for this section is primarily my own, and I have credited the contributions of others where appropriate.

The primary objective of this thesis is to offer a mathematical exposition of the fourth and fifth sections of Viazovska's original paper and to provide an account of the formalisation process. \todo{Say where we are with the formalisation before submitting.}