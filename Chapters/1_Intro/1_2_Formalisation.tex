\section{The Formalisation Movement}

While Hales announced his intent to formally verify his proof of the Kepler Conjecture in 2003, it was not till 2006, after Hales's solution appeared in the \textit{Annals}, that a formal description of Hales's formalisation project was published. Of his motivations, Hales wrote:
\begin{quote}
    \textit{In response to the lingering doubt about the correctness of the proof, at the beginning of 2003, I launched the \emph{Flyspeck} project, whose aim is a complete formal verification of the Kepler Conjecture. In truth, my motivations for the project are far more complex than a simple hope of removing residual doubt from the minds of few referees. Indeed, I see formal methods as fundamental to the long-term growth of mathematics.}~\cite{FlyspeckAnnouncement}
\end{quote}
Formal theorem proving was not unheard of in 2006. Interactive theorem provers, such as Coq and PRL, have existed since the 1980s. However, it was a relatively young field, and the amount of mathematics that had been formalised was limited. Hales's project was immensely ambitious, and the fact that it succeeded, despite taking over a decade, is impressive.

There is something prophetic about Hales's ``far more complex'' motivations for launching the Flyspeck project. The field of formal theorem proving has grown rapidly in the last decade, and interactive theorem provers like Lean are slowly making their way into mainstream mathematics. An excellent example of this is the formal verification of Gowers, Green, Manners and Tao's proof of Marton's Conjecture~\cite{PFRPublished}, which was formally verified in Lean in just three weeks. In particular, their proof was formally verified \textit{before} their paper was submitted for publication. The paper appeared in the \textit{Annals} in March 2025.

While this project has its similarities to Flyspeck, the objectives are slightly different. There is significant consensus in the mathematical community as to the correctness of Viazovska's result, and suspicions that $E_8$ is optimal in $\R^8$ existed long before her paper was published. The project is instead a challenge to the formalisation community---an attempt to push the capabilities of modern interactive theorem proving by formalising a Fields Medal-winning result mere years after its publication and sooner still after it was given this most prestigious recognition. While cutting-edge mathematics has been formalised \cite{liquidtensor} previously, formalising a result of the prestige, nuance, and beauty of Viazovska's would be a landmark achievement.

Associated to this project are a blueprint \cite{blueprint} and a \href{https://github.com/thefundamentaltheor3m/Sphere-Packing-Lean}{GitHub repository}. The first version of the blueprint was written by Viazovska herself, and read as a more detailed version of the original paper. Sections of the blueprint have been modified by Birkbeck, Hariharan, Lee and Ma, but the section describing the construction of the magic function, which will be the focus of this project, remains nearly identical to Viazovska's original blueprint.

There primary purpose of a blueprint is to offer a detailed exposition of the mathematics, reflecting a vision of the proof strategies to be used in the formalisation. One objective of this project is to offer a more detailed exposition still that can serve as an improvement of the blueprint that more closely resembles the actual state of the formalisation. The project blueprint was built using the Lean blueprint software \cite{Leanblueprint}, which has become an important part of modern, large-scale formalisation projects in Lean. It offers two extremely useful features: linking the definitions and theorems in the exposition to those in the code, and displaying the progress of the formalisation via a dependency graph. The dependency graph is colour-coded to reflect the state of the formalisation, and we invite the interested reader to view it \href{https://thefundamentaltheor3m.github.io/Sphere-Packing-Lean/blueprint/dep_graph_document.html}{here}.