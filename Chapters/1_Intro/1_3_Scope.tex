\section{The Scope of this Project}

% Ask how to word this... is "I" ok??

In November 2023, I had the privilege of meeting Maryna Viazovska while pursuing an exchange programme at the Swiss Federal Institute of Technology, Lausanne, where she is based. We began discussing formalising her solution to the sphere packing problem in $8$ dimensions, and soon initiated a collaboration with Christopher Birkbeck, Seewoo Lee, and Gareth Ma, with invaluable assistance from Kevin Buzzard, Utensil Song, and Patrick Massot. On 31 May 2024, Viazovska formally announced at the ICMS workshop \textit{Formalisation of Mathematics: Workshop for Women and Mathematicians of Minority Gender} that we would be attempting to formalise her groundbreaking paper.

Viazovska's original paper~\cite{Viazovska8} is divided into five sections. The first section introduces sphere packings and develops basic theory; the second discusses the Cohn-Elkies linear programming bounds~\cite[Theorem 3.1]{CohnElkies}; the third offers some background on the theory of modular forms; the fourth constructs two radial, Schwartz Fourier eigenfunctions with double zeroes at almost all points on the $E_8$ lattice; and finally, the fifth uses these eigenfunctions to construct the ``Magic Function'', a Schwartz function that satisfies the conditions of Cohn and Elkies's theorem to give an upper bound for all sphere packings in $\R^8$ that is equal to the density of the $E_8$ packing. The first two sections were formalised collaboratively in July and August 2024, and the third section is actively being worked on by Birkbeck and Lee. This project focuses on formalising the fourth and fifth sections of Viazovska's paper. The code written for this section is primarily my own, and I have credited the contributions of others where appropriate.

The primary objective of this thesis is to offer a mathematical exposition of the fourth and fifth sections of Viazovska's original paper and to provide an account of the formalisation process. \todo{Say where we are with the formalisation before submitting.}