\section{Preliminaries}

Before we begin defining things formally, we must include a small disclaimer about the terminology we have been using---and will continue to use---in this project. While \Cref{Ch1:Prob:SpherePacking_n} is usually referred to as the \textit{sphere} packing problem, a sphere is not usually thought to have an interior. Typically, in any metric space $X$ with metric $d$, the \textit{sphere} of radius $r \geq 0$ centred at $x \in X$ is defined to be $\setst{y \in X}{d(x, y) = r}$. In other words, the sphere consists only of a surface. In contrast, the sphere packing problem involves packing \textit{solid balls}. One can see why, in \cite{CannonHoney}, Hales opines that a more proper term for the problem would be the \textit{ball packing problem}. Nevertheless, in this project, we will continue to use the standard terminology, but we include this disclaimer so the reader bears in mind two things: first, that we will often mean `ball' when we use the word `sphere', and second, that we work with balls instead of spheres in Lean. We will also mention that it is convenient to require that the balls in question be open, so that the condition that spheres cannot overlap but merely touch tangentially can be shortened to that of disjointedness. We introduce notation.

% \begin{boxnotation}
%     For some $d \in \N$, $x \in \R^d$ and $r > 0$, we denote
%     \begin{align*}
%         B_d(x, r) := \setst{y \in \R^d}{\norm{x - y} < r}
%     \end{align*}
% \end{boxnotation}

We begin by defining a sphere packing. As we have stated, we want sphere packings to consist of disjoint spheres of the same radius. Given that lying on the interior of a certain sphere corresponds to being within some distance from its centre, we can capture this notion of disjointedness by imposing a separation condition on the set of centres of the sphere packing.