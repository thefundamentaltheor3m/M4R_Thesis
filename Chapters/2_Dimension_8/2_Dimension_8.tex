\chapter{The Ingredients of Viazovska's Solution}
\thispagestyle{empty}
% It might be worth chucking a large part of this chapter into an appendix. Specifically, §2.1.1 (Sphere Packing Fundamentals) and §2.3 (Modular Forms). Need to think about this... this would also depend on how we phrase §1.3 (scope of project) because we need to make it EXTREMELY CLEAR that the purpose of this project was NOT to delve deep into a cool application of the theory of modular forms but rather to learn how to formalise modern, computationally involved mathematics.

The purpose of this chapter is to offer background information that will be essential to understanding the rest of this exposition. We will begin by providing precise mathematical definitions for sphere packings, densities, and the sphere packing constant. We will then discuss the variation of the linear programming bound proven by Cohn and Elkies \cite[Theorem 3.1]{CohnElkies} used by Viazovska \cite[Theorem 2]{Viazovska8}. Finally, we will include a small discussion on the theory of modular forms and establish its relevance to the subsequent chapters of this thesis, which will focus on the construction of the Magic Function.

We will be minimalistic in our discussions, and primarily focus on motivating new concepts and their relevance to the sphere packing problem in dimension $8$. This section is not intended to offer an exhaustive treatment of the mathematics we will encounter, which is as vast as it is rich.

\section{Preliminaries}

Before we begin defining things formally, we must include a small disclaimer about the terminology we have been using---and will continue to use---in this project. While \Cref{Ch1:Prob:SpherePacking_n} is usually referred to as the \textit{sphere} packing problem, a sphere is not usually thought to have an interior. Typically, in any metric space $X$ with metric $d$, the \textit{sphere} of radius $r \geq 0$ centred at $x \in X$ is defined to be $\setst{y \in X}{d(x, y) = r}$. In other words, the sphere consists only of a surface. In contrast, the sphere packing problem involves packing \textit{solid balls}. One can see why, in \cite{CannonHoney}, Hales opines that a more proper term for the problem would be the \textit{ball packing problem}. Nevertheless, in this project, we will continue to use the standard terminology, but we include this disclaimer so the reader bears in mind two things: first, that we will often mean `ball' when we use the word `sphere', and second, that we work with balls instead of spheres in Lean. We will also mention that it is convenient to require that the balls in question be open, so that the condition that spheres cannot overlap but merely touch tangentially can be shortened to that of disjointedness. We introduce notation.

\begin{boxnotation}
    For some $d \in \N$, $x \in \R^d$ and $r > 0$, we denote
    \begin{align*}
        B_d(x, r) := \setst{y \in \R^d}{\norm{x - y} < r}
    \end{align*}
\end{boxnotation}

We organise this section into three subsections. The first defines fundamental notions about sphere packings. The second introduces the properties of two important, and closely related, classes of sphere packings, namely, lattice packings and periodic packings. The third subsection studies the most important sphere packing for our project: the $E_8$ lattice packing.

\subsection{Sphere Packing Fundamentals}

We begin by defining a sphere packing. As we have stated, we want sphere packings to consist of disjoint spheres of the same radius. Given that lying on the interior of a certain sphere corresponds to being within some distance from its centre, we can capture this notion of disjointedness by imposing a separation condition on the set of centres of the sphere packing.

\begin{boxdefinition}[Sphere Packing]
    Fix $d \in \N$ and $X \subset \R^d$. Assume that there exists a real number $r > 0$, known as the \textbf{separation radius}, such that
    \begin{align*}
        \norm{x - y} \geq r
    \end{align*}
    for all distinct $x, y \in X$. We define the \textbf{sphere packing with centres at $X$} to be
    \begin{align*}
        \Pa(X) := \bigcup_{x \in X} B_d(x, r)
    \end{align*}
\end{boxdefinition}

Note that the assumption that a separation radius exists is very important.

\begin{boxnexample}
    Let $d = 1$ and $X = \R$. Consider the set
    \begin{align*}
        \bigcup_{x \in \R} B_1(x, r) = \bigcup_{x \in \R} \parenth{x-r, x+r}
    \end{align*}
    For any $r > 0$, the above union is all of $\R$. However, it does not make sense to construct a sphere packing whose set of centres is the entirety of $\R$, as this would involve spheres overlapping. It is precisely to avoid such constructions that we impose the condition that $r$ be a separation radius on the set of centres.
\end{boxnexample}

Since all the information about a sphere packing is encoded in its set of centres and the corresponding separation radius (which must exist in order for the set of centres to be a valid set of centres for a sphere packing), we decided that a sphere packing would be formalised purely as a set of centres with a valid separation, and that a separate definition would be made to obtain the open balls that constitute the packing. We packaged the data of
\begin{itemize}
    \item the set of centres
    \item the separation radius
    \item the (automatically checked) condition that the separation radius is positive
    \item the condition that the set of centres is, indeed, separated by this radius
\end{itemize}
into a \verb|structure| called \verb|SpherePacking|: see \cite[\texttt{SpherePacking.Basic.SpherePacking}]{documentation}.\todo{Is this an acceptable way of citing the documentation? Should the repo be made public?}

We now define finite density, an indicator of how much of a bounded region of space a sphere packing covers.

\begin{boxdefinition}[Finite Density]\label{Ch2:Def:FiniteDensity}
    Let $\Pa$ be a sphere packing. For all $R > 0$, define the \textbf{finite density} to be
    \begin{align*}
        \Delta_\Pa(R) := \frac{\Volof{\Pa \cap B_d(0, R)}}{\Volof{B_d(0, R)}}
    \end{align*}
    where $\Vol$ is the Lebesgue measure on $\R^d$.
\end{boxdefinition}

Finite density is a somewhat local notion, in that it expresses sphere how closely packed spheres are in a bounded region. The sphere packing problem, on the other hand, examines the notion of closeness on a more global level. While taking the limit of finite densities as the radius of the bounding region approaches infinity might seem like a natural way to define density, it is not obvious that this limit always exists. Therefore, we define density to be the limit superior instead.

\begin{boxdefinition}[Density]\label{Ch2:Def:Density}
    Let $\Pa$ be a sphere packing. Define the \textbf{density} of $\Pa$ to be
    \begin{align*}
        \Delta(\Pa) := \limsup_{R \to \infty} \Delta_\Pa(R)
    \end{align*}
    where $\Delta_\Pa(R)$ is the finite density of $\Pa$, as defined in \Cref{Ch2:Def:FiniteDensity}.
\end{boxdefinition}

As one might expect, finite density and density are invariant under scaling.

\begin{boxproposition}\label{Ch2:Prop:Scaling_Sphere_Packings}
    Let $\Pa$ be a sphere packing. Fix $\lambda > 0$. Denoting by $\lambda \Pa$ the sphere packing obtained by scaling the spheres and the set of centres in $\Pa$ by a factor of $\lambda$, we have
    \begin{align*}
        \Delta_{\Pa}\of{R} = \Delta_{\lambda \Pa}\of{\lambda R}
    \end{align*}
    for all $R > 0$. Similarly, we have
    \begin{align*}
        \Delta\of{\Pa} = \Delta\of{\lambda \Pa}
    \end{align*}
\end{boxproposition}

The sphere packing problem asks for the sphere packing that achieves the highest possible density. We can be formal about the notion of the highest possible density.

\begin{boxdefinition}[Sphere Packing Constant]
    The \textbf{sphere packing constant} in $\R^d$, for any $d > 0$, is defined to be
    \begin{align*}
        \Delta_d := \sup\!\parenth{{\setst{\Delta_{\Pa}}{\Pa \text{ is a sphere packing in } \R^d}}}
    \end{align*}
\end{boxdefinition}

\Cref{Ch2:Prop:Scaling_Sphere_Packings} tells us that it suffices to take the supremum over sphere packings of separation $1$, because a sphere packing $\Pa$ can be scaled down by its separation radius to obtain a sphere packing of separation $1$.

\begin{boxproposition}\label{Ch2:Prop:Scaling_Sphere_Packing_Constant}
    For all $d$, we have
    \begin{align*}
        \Delta_d = \sup\!\parenth{{\setst{\Delta_{\Pa}}{\Pa \text{ is a sphere packing in } \R^d \text{ with separation } 1}}}
    \end{align*}
\end{boxproposition}

As intuitive as this result might seem, we do mention it here because it is something we have to deal with explicitly when formalising the theory of sphere packings in Lean. The first instance when we really see it in action is the proof of \Cref{SP:Thm:CohnElkies}.

The objective of the sphere packing problem in any dimension $d$ is to optimise the sphere packing constant $\Delta_d$. As we have seen, this is a highly non-trivial thing to do. We can, however, offer a trivial upper-bound on sphere packing density.

\begin{boxlemma}
    For any sphere packing $\Pa$ and $R > 0$, we have that $\Delta_{\Pa}(R) \leq 1$.
\end{boxlemma}
\begin{proof}
    This is an immediate consequence of the fact that $\Pa \cap B_d(0, R) \subseteq B_d(0, R)$.
\end{proof}
This immediately gives us the following basic facts.
\begin{boxcorollary}
    For any sphere packing $\Pa$, we have that $\Delta_{\Pa} \leq 1$.
\end{boxcorollary}
\begin{boxcorollary}
    For any $d \in \N$, $\Delta_d \leq 1$.
\end{boxcorollary}

This is not a very good upper-bound. However, it tells us that the sphere packing constant in any number of dimensions is a finite real number in the interval $(0, 1]$. There is still some work to be done before we can give better bounds on the sphere packing constant. Furthermore, it is unclear whether the sphere packing constant actually is, for a general $d$, the density of a sphere packing in $\R^d$. Nevertheless, this is a good starting point.

A great deal of basic sphere packing API in Lean was developed in July 2024 for the project to formalise Viazovska's solution in dimensino $8$. The majority of the code was written by Gareth Ma, who also made significant improvements to the design choices I had made when setting up the project. The definitions and results in this section have all been formalised, and information about the code that has been written can be found in the project documentation \cite[\texttt{SpherePacking.Basic.SpherePacking}]{documentation}.

In the next subsection, we discuss a special class of sphere packings that have periodicity properties with respect to lattices.

\subsection{Lattice and Periodic Sphere Packings}

We begin by defining lattices and briefly commenting on existing \verb|Mathlib| API on lattices. There are primarily two ways in which lattices are defined in mathematical literature. A lattice in some Euclidean space $\R^n$ is either described as the $\Z$-span of some $\R$-basis of $\R^n$ or as a discrete, co-compact subgroup. One can borrow characteristics from both definitions to construct other equivalent definitions.

The characteristics described in both definitions do exist \verb|Mathlib|. However, given that one of the objectives of creating a unified mathematics library is centralisation, a combination of these definitions is used as the \textit{definition} of a \verb|class| that we call \verb|IsZLattice| and information about its many properties, as well as the $\Z$-span construction, are encoded in theorems. In particular, we have a theorem that tell us that every lattice is a free $\Z$-submodule, meaning it has a $\Z$-basis, and that this $\Z$-basis is actually an $\R$-basis of the ambient space. Furthermore, we have a result that every object constructed in that manner is a lattice. Results about types bearing the \verb|IsZLattice| class (ie, lattices as they are defined in \verb|Mathlib|) live in the \verb|ZLattice| namespace, whereas results bout $\Z$-spans of $\R$-bases live in the \verb|ZSpan| namespace. \todo{Create citation for Mathlib.}

We begin by stating the \verb|Mathlib| definition of a lattice.

\begin{boxdefinition}[Lattice]
    A \textbf{lattice} in a Euclidean space $\R^n$ is a discrete $\Z$-submodule of $\R^n$ such that its $\R$-span contains every element in $\R^n$.
\end{boxdefinition}

The \verb|Mathlib| definition is more general, and works for any normed vector space over a normed field. Here, the word `discrete' means that the lattice is discrete in a topological sense, meaning that the subspace topology on the lattice is precisely the discrete topology.

\begin{boxdefinition}[Periodic Sphere Packing]
    Let $\Lambda \subset \R^d$ be a lattice. We say a sphere packing $\Pa(X)$ with spheres centred at points in $X \subset \R^d$ is \textbf{periodic with respect to $\Lambda$}, or \textbf{$\Lambda$-periodic},
    \begin{align*}
        \lambda + X = X
    \end{align*}
    ie, for all $\lambda \in \Lambda$ and $x \in X$, we have that $\lambda + x \in X$.
\end{boxdefinition}

We define Periodic Sphere Packings in Lean as extending the definition of Sphere Packings by creating a \verb|structure| called \verb|PeriodicSpherePacking| that packages the additional data of
\begin{itemize}
    \item the lattice, viewed as a $\Z$-submodule of the ambient space $\R^d$
    \item the condition that the set of centres is periodic with respect to this $\Z$-submodule
    \item the (automatically checked) condition\footnotemark{} that the $\Z$-submodule is discrete
    \item the (automatically checked) condition\footnotemark[\value{footnote}] that the discrete $\Z$-submodule is a lattice
\end{itemize}
\footnotetext{more precisely, the automatically inferred instance}
The definition is in \cite[\texttt{SpherePacking.Basic.SpherePacking}]{documentation}.

Lattice packings are a special class of periodic packings.

\begin{boxdefinition}[Lattice Packing]
    Let $\Lambda \subset \R^d$ be a lattice. The \textbf{$\Lambda$ lattice packing} is the sphere packing with centres at points in $\Lambda$. Such a sphere packing admits a separation radius because $\Lambda$ is discrete and is $\Lambda$-periodic because $\Lambda$ is closed under addition.
\end{boxdefinition}

In \Cref{Ch2:Subsec:E8}, we will briefly examine a specific lattice packing, the $E_8$ lattice packing.

The periodicity property of a periodic sphere packing can be exploited to derive a more convenient formula for its density.

\begin{boxproposition}\label{Ch2:Prop:Periodic_Density}
    Let $\Pa(X)$ be a sphere packing with centres at $X \subset \R^d$ and separation $r$ that is periodic with respect to some lattice $\Lambda \subset \R^d$. We have that
    \begin{align}
        \Delta_{\Pa(X)} = \abs{\quotient{X}{\Lambda}} \frac{\Volof{B_d\of{0, \frac{r}{2}}}}{\Volof{\quotient{\R^d}{\Lambda}}}
        \label{Ch2:Eq:Periodic_Density}
    \end{align}
\end{boxproposition}

The proof is beyond the scope of this M4R project, but was formalised in Summer 2024: see \verb|PeriodicSpherePacking.density_eq'| in \cite[\texttt{SpherePacking.Basic.PeriodicPacking}]{documentation}.

Just as we defined the sphere packing constant for any dimension $d \in \N$, we can define a \textit{periodic} sphere packing constant in any dimension.

\begin{boxdefinition}[Periodic Sphere Packing Constant]
    For all $d \in \N$, define the \textbf{periodic sphere packing constant in dimension $d$} to be
    \begin{align*}
        \Delta_{d}^{\periodic} = \sup\of{{\setst{\Delta_{\Pa}}{\Pa \text{ is a periodic sphere packing in } \R^d}}}
    \end{align*}
\end{boxdefinition}

The power of periodic sphere packings is illustrated by a rather surprising fact.

\begin{boxproposition}\label{Ch2:Prop:Periodic_Const_eq_Const}
    For all $d \in \N$,
    \begin{align*}
        \Delta_{d} = \Delta_{d}^{\periodic}
    \end{align*}
\end{boxproposition}

We do not prove this result here, as it is beyond the scope of this M4R. A proof can be found in \cite[Appendix A]{CohnElkies}. We do, however, mention that \Cref{Ch2:Prop:Periodic_Const_eq_Const} can be combined with \Cref{Ch2:Prop:Scaling_Sphere_Packing_Constant} to give the following.

\begin{boxproposition}\label{Ch2:Prop:Scaling_Periodic_Constant}
    For all $d \in \N$,
    \begin{align*}
        \Delta_{d} = \sup\of{{\setst{\Delta_{\Pa}}{\Pa \text{ is a periodic sphere packing in } \R^d \text{ with separation } 1}}}
    \end{align*}
\end{boxproposition}

\Cref{Ch2:Prop:Periodic_Const_eq_Const} tells us that finding a sphere packing that satisfies the \textit{periodic} sphere packing constant gives us the optimal sphere packing in dimension $d$. \Cref{Ch2:Prop:Scaling_Periodic_Constant} allows us to focus our search even more. We will exploit these two facts in \Cref{Ch2:Sec:CohnElkies}, where we will construct an upper bound for all sphere packing densities in dimension $d$ by constructing an upper-bound for the periodic sphere packing constant in dimension $d$. When constructing this upper-bound, we will exploit the fact that periodic sphere packings admit a `nice' density formula (cf. \Cref{Ch2:Prop:Periodic_Density}). The results we have seen about periodic sphere packings will thus greatly simplify our task of finding the optimal sphere packing in dimension $8$.

We are now ready to discuss a special sphere packing in $\R^8$: the $E_8$ sphere packing.

\subsection{The $E_8$ Lattice Packing}\label{Ch2:Subsec:E8}

\begin{wrapfigure}[18]{r}{0.5\linewidth}
    \vspace{-3em}
    \centering
    \includegraphics[width=0.98\linewidth]{Chapters/2_Dimension_8/Images/Gorbe_E8.png}
    \caption{The Coxeter projection of the $E_8$ root system. \cite{Gorbe_E8}}
    \label{Ch2:Fig:Gorbe_E8}
\end{wrapfigure}

It is quite remarkable that $E_8$ should show up when discussing sphere packings. At its core, $E_8$ is an irreducible root system. It shows up in the classification of important classes of objects like irreducible Coxeter groups, crystallographic Coxeter groups, and semi-simple Lie algebras over $\C$. $E_8$ is not a classical root system but an \textit{exceptional} root system, meaning that the geometric properties of its roots cannot be found in irreducible root systems in all dimensions.

The $E_8$ root system consists of $240$ vectors in $\R^8$ that are permuted by a certain finite subgroup of the $8$-dimensional orthogonal group. This group is sometimes referred to as the $E_8$ Coxeter group or as the Weyl group of the $E_8$ lattice. These roots can be divided into $8$ orbits, each of which corresponds to one of the `layers' of concentric circles in \Cref{Ch2:Fig:Gorbe_E8}. The dots in the figure correspond to projections of the roots onto a plane on which a specific type of element of the Coxeter group, known as a Coxeter element, acts as a rotation. This visualisation offers a convenient---and aesthetically pleasing---means of visualising this collection of $8$-dimensional vectors and appreciating some of its symmetry.

% From a motivational standpoint, this is quite important. So mention facts like distances between lattice points and things like that. Important for construction of magic function.



\section{The Cohn-Elkies Linear Programming Bounds}
\section{A Word on Modular Forms}
\label{Ch2:Sec:ModForms}
% SHOULD I MAYBE TURN THIS INTO AN APPENDIX? IT'S LOOKING PRETTY LONG...

\begin{comment}
Things to discuss:
1. What is a modular form
2. What is a quasimodular form
3. Examples: Eisenstein Series, Jacobi Theta functions, Discriminant form
We can reference things like the q-expansions of the Eisenstein series, the transformation rules for the Jacobi theta functions, and the product formula for the discriminant form.
\end{comment}

In this section, we give a brief introduction to the theory of modular forms. Birkbeck, Loeffler and others have formalised several results in the theory of modular forms, and a significant portion of their work has been merged into \mathlib. Definitions and results from this section that pertain to Viazovska's solution to the sphere packing problem in $\R^8$ that do not feature in \mathlib\ are being actively formalised by Birkbeck and Lee, with contributions from Ma.

First, we introduce the following useful notation.

\begin{boxnotation}
    For the remainder of this paper, denote the Complex upper half-plane by $\Halfplane$. That is, define $\Halfplane := \setst{z \in \C}{0 < \Im(z)}$.
\end{boxnotation}

This corresponds to the \mathlib\ notation for the upper half-plane.

A key motivating idea in the study of modular forms is the study of the action of $\SL{2, \Z}$ on $\Halfplane$ by Möbius transformations via
\begin{align*}
    \begin{bmatrix}
        a & b \\ c & d
    \end{bmatrix}
    \cdot z := \frac{az + b}{cz + d}
\end{align*}
That matrix multiplication corresponds to the composition of Möbius transformations is a well-known fact in Complex Analysis. One can hence show that the above is indeed a group action.

Both the identity $I \in \SL{2, \Z}$ and the negative identity $-I \in \SL{2, \Z}$ have the same (trivial) action on $\Halfplane$. Indeed, the $\SL{2, \Z}$ action descends to a faithful action of $\PSL{2, \Z} = \quotient{\SL{2, \Z}}{\set{\pm I}}$ on $\Halfplane$. Since we are more interested in the \textit{actions} of matrices in $\SL{2, \Z}$ and $\PSL{2, \Z}$ than we are in their entries, we often do not distinguish between the two groups.

The \mathlib\ definition of a modular form is more general than the first definitions of modular forms often seen in literature (see \cite[Chapter VII, \S 2, Definition 4]{SerreArith} and \cite[Definition 1.1.2]{DiamondShurman}), and instead matches subsequent definitions that generalise these first definitions. Modular forms are usually described as functions that are holomorphic on the upper half-plane that are invariant under the $\SL{2, \Z}$-action up to an \textit{automorphy factor} of a certain \textit{weight}. This \textit{weight} is defined as the \textit{weight of the modular form}. However, one is often interested in invariance under not all of $\SL{2, \Z}$, but certain \textit{principal congruence subgroups} or subgroups containing such subgroups, known as \textit{congruence subgroups}. Each principal congruence subgroup has a \textit{level}, which is defined to be the \textit{level} of a modular form whose congruence subgroup is principal. Modular forms, as they are defined in \mathlib\ and in the blueprint \cite{blueprint}, are therefore indexed by two properties: a \textit{congruence subgroup of $\SL{2, \Z}$}, which indicates the scope of invariance under the $\SL{2, \Z}$-action, and a \textit{weight}, which gives the extent of invariance under the action of elements of the subgroup in question.

To give a complete definition of the weight of a modular form, we need to define automorphy factors and the slash action notation.

\begin{boxdefinition}[Automorphy Factors and Slash Actions]\label{Ch2:Def:Aut_Factor_Slash_Action}
    Fix $k \in \Z$, $z \in \Halfplane$ and $\gamma = \begin{bmatrix} a & b \\ c & d \end{bmatrix} \in \SL{2, \Z}$. Define the \textbf{automorphy factor of weight $k$} to be
    \begin{align}
        j_k\of{z, \gamma} &:= \parenth{cz + d}^{-k}
        \label{Ch2:Eq:AutomorphyFactor_def}
    \end{align}
    For any function $f : \Halfplane \to \C$, with $k$ and $\gamma$ as above, the \textbf{slash operator} maps $f$ to a new function $f \mid_k \gamma : \Halfplane \to \C$ given by
    \begin{align}
        \fmof{k}{\gamma}{z} &:= j_k\of{z, \gamma} \fof{\gamma \cdot z} = \parenth{cz + d}^{-k} \fof{\frac{az + b}{cz + d}}
        \label{Ch2:Eq:SlashAction_def}
    \end{align}
    The action of $\gamma$ mapping $f$ to $\fm_k \gamma$ via the weight $k$ slash operator is sometimes referred to as a \textbf{slash action}.
\end{boxdefinition}

It is clear, from the above definition, that $\fm_0 \gamma = f \circ \gamma$ for al $\gamma \in \SL{2, \Z}$. That is, if $f = \fm_0 \gamma$, then $f = f \circ \gamma$, that is, $f$ is invariant under composition with (the action of) $\gamma$. If $f = \fm_k \gamma$ for some $k \in \Z$ and $\gamma \in \SL{2, \Z}$, we can view the weight $k$ as indicating the `extent of invariance' of $f$ under composition with $\gamma$. Note that slash actions can be composed.

\begin{boxlemma}\label{Ch2:Lemma:Slash_Mul}
    For all $k \in \Z$, $f : \Halfplane \to \C$ and $\gamma_1, \gamma_2 \in \SL{2, \Z}$,
    \begin{align*}
        \parenth{f \mid_k \gamma_1} \mid_k \gamma_2 = f \mid_k \parenth{\gamma_1 \gamma_2}
    \end{align*}
    where $\gamma_1 \gamma_2$ is the product of $\gamma_1$ and $\gamma_2$ as matrices.
\end{boxlemma}
A slightly more general version of this has been \href{https://github.com/leanprover-community/mathlib4/blob/a0370507e2922f0a329a2d8cc17e9f9148cd168d/Mathlib/NumberTheory/ModularForms/SlashActions.lean#L77}{formalised} in \mathlib.

We are now ready to define congruence subgroups, which will tell us under precisely which elements of $\SL{2, \Z}$ a modular form is slash-invariant. We express this notion in the language of modular arithmetic.

\begin{boxdefinition}[Congruence Subgroup]\label{Ch2:Def:Cong_Subgroup}
    Fix $N \in \N$. The \textbf{level $N$ principal congruence subgroup} of $\SL{2, \Z}$, denoted $\Gamma(N)$, is defined to be the kernel of the surjective group homomorphism from $\SL{2, \Z}$ to $\SL{2, \Zmod{N}}$ that comes from reducing modulo $N$. That is,
    \begin{align}
        \Gamma(N) &:= \setst{
        \begin{bmatrix}
            a & b \\ c & d
        \end{bmatrix} \in \SL{2, \Z}}{
        \begin{bmatrix}
            a & b \\ c & d
        \end{bmatrix}
        \equiv
        \begin{bmatrix}
            1 & 0 \\ 0 & 1
        \end{bmatrix}
        \pmod{N}}
        \label{Ch2:Eq:PrincipalCongruenceSubgroup_def}
    \end{align}
    More generally, a subgroup $\Gamma$ of $\SL{2, \Z}$ is called a \textbf{congruence subgroup} if $\Gamma(N) \subset \Gamma$ for some $N \in \N$.
\end{boxdefinition}

We now have enough to define what it means for a holomorphic function to be invariant under the slash action of a congruence subgroup. In the definition of modular forms, however, we include an additional condition that is often referred to as \textit{holomorphicity at $i\infty$}, the purpose of which is to ensure that spaces of modular forms, which turn out to admit $\C$-vector space structures, are, in fact, finite-dimensional \cite{KevinFamilies}.

The theory of modular forms is often thought to lie in the very rich intersection of algebra and analysis. Our definitions so far have been largely algebraic, but our next one is analytic. Consider the mapping $q : \Halfplane \to \C : z \mapsto e^{2\pi i z}$. This maps $\Halfplane$ to the punctured, open unit disc
\begin{align*}
    D := \setst{w \in \C}{0 < \abs{q} < 1}
\end{align*}
Indeed, for all $z \in \Halfplane$, writing $z = x + iy$ for $x, y \in \R$ with $y > 0$, we have
\begin{align*}
    \abs{q(z)} = \abs{e^{2 \pi i \parenth{x + iy}}} = \abs{e^{2\pi i x}} \cdot \abs{e^{-2\pi y}} < 1
\end{align*}
with $0 \notin q\of{\Halfplane}$ but $q(z) \to 0$ as $\Im(z) = y \to \infty$. Now, we know that the holomorphic functions from $D \to \C$ are precisely those that have Laurent expansions of the form
\begin{align*}
    \sum_{n=0}^{\infty} c_n w^n
\end{align*}
for all $w \in D$. If we write $w = q(z)$ for $z \in \Halfplane$, the above series turns out to be a \textit{Fourier expansion}. We can hence make the following definition for holomorphicity at $i\infty$.

\begin{boxdefinition}[Holomorphicity at $i\infty$]\label{Ch2:Def:Holo_at_ImInfty}
    We say a function $f : \Halfplane \to \C$ is \textbf{holomorphic at $i\infty$} if $f$ admits a Fourier expansion of the form
    \begin{align*}
        f(z) = \sum_{n=0}^{\infty} c_n q(z)^n = \sum_{n=0}^{\infty} c_n e^{2\pi i nz}
    \end{align*}
    That is, $f$ admits a Fourier expansion with no negative powers of $q(z)$.
\end{boxdefinition}

The holomorphicity of $f$ at $i\infty$ essentially means that the Fourier expansion of $f$ is a holomorphic $D \to \C$ function in $q(z)$, with the added constraint that $\abs{f(z)}$ remains bounded as $\Im(z) \to \infty$, that is, the corresponding $D \to \C$ function in $q(z)$ extends to a holomorphic function that is defined and bounded at $0$. There is a rich theory of functions where $c_0 = 0$, but we will not explore that theory here.\footnote{Modular forms with this property are known as \textbf{cusp forms}. One modular form we will need to construct the magic function is the discriminant form, which will turn out to be a cusp form.}

We are now ready to define modular forms. Intuitively, a modular form is a function that satisfies the above definitions in a slash-invariant manner. More precisely, we have the following.

\begin{boxdefinition}[Modular Form]
    Fix $k \in \Z$ and let $\Gamma$ be a congruence subgroup of $\SL{2, \Z}$. We say a function $f : \Halfplane \to \C$ is a \textbf{modular form of weight $k$ with respect to $\Gamma$} if $f$ is \textbf{invariant} under the slash action of $\Gamma$ and \textbf{holomorphic at $i\infty$} under the slash action of $\SL{2, \Z}$. That is,
    \begin{enumerate}
        \item For all $\gamma \in \Gamma$, $\fm_k \gamma = f$ (cf. \Cref{Ch2:Def:Aut_Factor_Slash_Action}).
        \item For all $\gamma \in \SL{2, \Z}$, $\fm_k \gamma$ is holomorphic at $i\infty$ (cf. \Cref{Ch2:Def:Holo_at_ImInfty}).
    \end{enumerate}
    We denote by $M_k(\Gamma)$ the space of modular forms of weight $k$ and congruence subgroup $\Gamma$. If $\Gamma = \Gamma(N)$ for some $N \in \N$, we say an element of $M_k(\Gamma)$ has \textbf{level $N$}.
\end{boxdefinition}

There is an immensely rich theory of modular forms, and for the purposes of practicality, it was decided not to explore this theory in great detail in this project, particularly because the formalisation of the aspects of Viazovska's proof that stem from this theory is being led by Birkbeck, Lee and Ma. We will instead use the remainder of this section to discuss three specific (families of) modular forms and those of their properties that Viazovska uses to construct her magic function.

\subsection{The Eisenstein Series}
\label{Ch2:Subsec:EisensteinSeries}

The Eisenstein Series are an important family of slash-invariant forms that will prove essential to the construction of the magic function. The Eisenstein Series whose \textit{weight} is an even integer that is at least $4$ are modular forms, though we will also need to work with the Eisenstein Series of weight $2$, which, despite not being a modular form, is sufficiently well-behaved for our purposes. We will define it separately from those Eisenstein Series that are modular forms.

Let $k \geq 4$ be an even integer. We denote by $E_k$ the weight $k$ Eisenstein Series. There is more than one way to define $E_k$. In this report, we give the definition that was formalised by Birkbeck for this project. Birkbeck's definition in the project repository is a particular case of the \href{https://github.com/leanprover-community/mathlib4/blob/70816aec3a0f7bb98ac42991652a66b6405e1a00/Mathlib/NumberTheory/ModularForms/EisensteinSeries/Basic.lean#L28-L35}{\mathlib\ definition}, which defines it as a \texttt{ModularForm} structure combining the function \href{https://github.com/leanprover-community/mathlib4/blob/70816aec3a0f7bb98ac42991652a66b6405e1a00/Mathlib/NumberTheory/ModularForms/EisensteinSeries/Defs.lean#L107}{\texttt{eisensteinSeries}} with the proofs of the properties that make it a modular form. The \mathlib\ definition is more general than the one we study here, and involves imposing congruence conditions on the subsets of the lattice $\Z^2$ over which the Eisenstein Series are summed. It is not necessary for this project.

\begin{boxdefinition}[The Eisenstein Series of Even Weight $\geq 4$]\label{Ch2:Def:EisensteinSeries_geq_4}
    For $k \geq 4$ even, define the \textbf{weight $k$ Eisenstein Series} to be the function $E_k : \Halfplane \to \C$ given by
    \begin{align}
        E_k(z) := \frac{1}{2} \sum_{\substack{\parenth{m, n} \in \Z^2 \\ \gcd(m, n) = 1}} \frac{1}{\parenth{mz + n}^k}
        \label{Ch2:Eq:Eisenstein_def_Chris}
    \end{align}
    with the defining summation converging absolutely.
\end{boxdefinition}

Note that the Eisenstein Series can also be defined as
\begin{align}
    E_k(z) = \frac{1}{2\zeta(k)} \sum_{\parenth{m, n} \in \Z^2 \setminus \set{0}} \frac{1}{\parenth{mz + n}^k}
    \label{Ch2:Eq:Eisenstein_def_with_zeta_normalisation}
\end{align}
with $\zeta$ here denoting the Riemann zeta function. It is shown in \cite[Equation (4.1), pp. 109-110]{DiamondShurman} that this definition matches the definition formalised by Birkbeck in the project repository and stated informally in \Cref{Ch2:Def:EisensteinSeries_geq_4}.\todo{Blueprint gives zeta def whereas repo gives coprime def. WHOOPSIE!}

It is shown in \cite[pp. 4-5]{DiamondShurman} that $E_k$ is a weight $k$, level $1$ modular form for even integers $k \geq 4$. That is, $E_k$ is invariant under the weight $k$ slash-action of every element of $\SL{2, \Z}$. As important special cases of this, $E_k$ satisfies two important functional equations.

\begin{boxproposition}\label{Ch2:Prop:Eisenstein_func_eq}
    For all even $k \geq 4$ and $z \in \Halfplane$, the following both hold:
    \begin{align}
        E_k\of{z + 1} &= E_k \label{Ch2:Eq:Ek_func_eq_one_add} \\
        E_k\of{-\frac{1}{z}} &= z^k E_k(z) \label{Ch2:Eq:Ek_func_eq_neg_one_div}
    \end{align}
\end{boxproposition}
\begin{proof}
    Both of these are just slash-invariance properties in disguise. We have\todo{Fix slash formatting}
    \begin{align*}
        E_k(z + 1) = \parenth{E_k \; \middle\vert_k \; {\begin{bmatrix} 1 & 1 \\ 0 & 1 \end{bmatrix}}}\of{z} = \parenth{0z + 1}^{k} E_k(z) = E_k(z)
    \end{align*}
    Similarly, we have
    \begin{align*}
        E_k\of{-\frac{1}{z}} = \parenth{E_k \; \middle\vert_k \; {\begin{bmatrix} 0 & -1 \\ 1 & 0 \end{bmatrix}}}\of{z} = \parenth{1z + 0}^k E_k(z) = z^k E_k(z)
    \end{align*}
    as required.
\end{proof}

The functional equations \eqref{Ch2:Eq:Ek_func_eq_one_add} and \eqref{Ch2:Eq:Ek_func_eq_neg_one_div} yield similar results for an important function that will be used in constructing the magic function. We will explore this idea in \Cref{Ch4:Chapter}.

One of the most important properties of the Eisenstein Series---at least, for our purposes---is that their Fourier coefficients\footnote{The slash-invariant properties of modular forms mean that they have periodicity properties. Computing their Fourier series is hence a natural strategy when attempting to dissect their properties.} grow polynomially. We will be particularly interested in $E_4$ and $E_6$, which are defined as above, and their cousin $E_2$, which we will treat separately. These functions show up in the definition of Viazovska's magic function, and the polynomial growth property allows us to prove that the magic function is Schwartz.

Our strategy to prove that the Fourier coefficients have polynomial growth will be to compute them explicitly. First, we need to define the arithmetic function $\sigma_k(n)$, which is defined in \mathlib\ as \href{https://github.com/leanprover-community/mathlib4/blob/70816aec3a0f7bb98ac42991652a66b6405e1a00/Mathlib/NumberTheory/ArithmeticFunction.lean#L797-L799}{\texttt{ArithmeticFunction.sigma}}.

\begin{boxdefinition}[The $\sigma$-Function]
    The \textbf{$\sigma$-function} $\sigma : \N \times \N \to \N$ is given by
    \begin{align*}
        \sigma_k\of{n} := \sum_{d \mid n} d^k
    \end{align*}
\end{boxdefinition}

In \mathlib, for every natural number \texttt{k}, \texttt{ArithmeticFunction.sigma k} is defined as an \href{https://github.com/leanprover-community/mathlib4/blob/bc10be4a66942c0fc2547b54f7f8715df72ff28c/Mathlib/NumberTheory/ArithmeticFunction.lean#L76-L80}{\texttt{ArithmeticFunction $\N$}} structure, meaning it is an $\N \to \N$ map that maps $0$ to $0$.

The reason we defined the $\sigma$-function is that the Fourier coefficients of the Eisenstein series are given in terms of $\sigma$.

\begin{boxtheorem}\label{Ch2:Thm:Ek_qexpansion}
    For all even $k \geq 4$ and $z \in \Halfplane$, $E_k(z)$ can be expressed as the Fourier series
    \begin{align}
        E_k(z) &= 1 + C_k \sum_{n=1}^{\infty} \sigma_{k-1}\of{n} e^{2\pi i n z}
        \label{Ch2:Eq:Ek_qexpansion}
    \end{align}
    where
    \begin{align}
        C_k = \frac{1}{\zeta(k)} \cdot \frac{\parenth{-2 \pi i}^k}{\parenth{k-1}!}
        \label{Ch2:Eq:Ck_Ek_qexpansion_const}
    \end{align}
    In particular, $C_4 = 240$ and $C_6 = -504$. That is, $E_4$ and $E_6$ have the following Fourier expansions:
    \begin{align}
        E_4(z) &= 1 + 240 \sum_{n=1}^{\infty} \sigma_3(n) e^{2 \pi i n z} \label{Ch2:Eq:E4_qexpansion} \\
        E_6(z) &= 1 - 504 \sum_{n=1}^{\infty} \sigma_5(n) e^{2 \pi i n z} \label{Ch2:Eq:E6_qexpansion}
    \end{align}
\end{boxtheorem}

The statement and proof of the general Fourier expansion of $E_k$ for even $k \geq 4$ have been \href{https://github.com/thefundamentaltheor3m/Sphere-Packing-Lean/blob/076f4b8d6a37fa95de3bc4764a5d7f911fde91e0/SpherePacking/ModularForms/Eisensteinqexpansions.lean#L301}{formalised by Birkbeck} in the Sphere Packing repository.\todo{Here, again, the repo disagrees with the blueprint. FIX!} Substituting $k = 4$ and $k = 6$ in the expression for $C_k$ and evaluating it using software like Wolfram|Alpha gives the desired result.

Now, it is immediate that the Fourier coefficients exhibit polynomial growth: for all $k, n \in \N$, $\sigma_k(n)$ is a sum of at most $n$ numbers that are each at most $n^k$, meaning $\sigma_k(n) \leq n^{k+1}$.

% TODO: RAISE THIS ON ZULIP. Mention that we need to correct the blueprint to reflect the repo version.

For the remainder of this subsection, we will focus on a cousin of the weight $\geq 4$ Eisenstein Series: the weight $2$ Eisenstein Series, denoted $E_2$. The reason why we treat $E_2$ separately is that it is not a modular form. Furthermore, it cannot be defined via the summation used in \Cref{Ch2:Eq:Eisenstein_def_Chris} or \Cref{Ch2:Eq:Eisenstein_def_with_zeta_normalisation}: unfortunately, when $k = 2$, these sums do not converge absolutely. That being said, Birkbeck has \href{https://github.com/thefundamentaltheor3m/Sphere-Packing-Lean/blob/076f4b8d6a37fa95de3bc4764a5d7f911fde91e0/SpherePacking/ModularForms/summable_lems.lean#L1680}{shown formally} that for all $m \in \Z$, $z \in \Halfplane$, and $k \geq 2$, the summation
\begin{align*}
    \sum_{n \in \Z} \frac{1}{\parenth{mz + n}^k}
\end{align*}
converges absolutely. He then shows, through several \sorry-free lemmas, that
\begin{align*}
    \lim_{N \to \infty} \sum_{m = -N}^{N - 1} \sum_{n \in \Z} \frac{1}{\parenth{mz + n}^k}
\end{align*}
exists, allowing us to define $E_2$ in the following manner.

\begin{boxdefinition}[$E_2$]
    For all $z \in \Halfplane$, define
    \begin{align*}
        E_2(z) := \frac{1}{2\zeta(2)} \lim_{N \to \infty} \sum_{m = -N}^{N - 1} \sum_{n \in \Z} \frac{1}{\parenth{mz + n}^k}
    \end{align*}
\end{boxdefinition}

The difference between this definition and \eqref{Ch2:Eq:Eisenstein_def_with_zeta_normalisation} with $k = 2$ is that here, we specify an order of summation for the outer sum, whereas for $k \geq 4$, in both \eqref{Ch2:Eq:Eisenstein_def_with_zeta_normalisation} and \eqref{Ch2:Eq:Eisenstein_def_Chris}, the order is immaterial due to absolute convergence. Interestingly, the Fourier expansion of $E_2$ agrees with \eqref{Ch2:Eq:Ek_qexpansion}.

\begin{boxtheorem}\label{Ch2:Thm:E2_qexpansion}
    For all $z \in \Halfplane$, $E_2(z)$ can be expressed as the Fourier series
    \begin{align}
        E_2(z) = 1 - 24 \sum_{n=1}^{\infty} \sigma_1(n) e^{2 \pi i n z}
        \label{Ch2:Eq:E2_qexpansion}
    \end{align}
\end{boxtheorem}

Birkbeck gives a formal proof of this over the course of several \sorry-free lemmas in \href{https://github.com/thefundamentaltheor3m/Sphere-Packing-Lean/blob/076f4b8d6a37fa95de3bc4764a5d7f911fde91e0/SpherePacking/ModularForms/E2.lean#L736}{\texttt{SpherePacking.ModularForms.E2.lean}}. Interestingly, substituting $k = 2$ in \eqref{Ch2:Eq:Ck_Ek_qexpansion_const} yields precisely $-24$. Moreover, the same argument we used earlier demonstrates that the Fourier coefficients of $E_2$ also grow polynomially. We will mention this result again in \Cref{Ch4:Chapter}, where we will prove that the magic function is Schwartz.

 We end our discussion on the Eisenstein Series by giving an explicit counterexample to weight $2$, level $1$ slash-invariance that shows that $E_2$ is not a weight $2$, level $1$ modular form.

\begin{boxlemma}\label{Ch2:Lemma:E2_slash_action}
    For all $\gamma = \begin{bmatrix} a & b \\ c & d \end{bmatrix} \in \SL{2, \Z}$, we have
    \begin{align*}
        E_2 \mid_2 \gamma = \parenth{cz + d}^{-2} E_2\of{\frac{az + b}{cz + d}} = E_2(z) - \frac{6ic}{\pi\parenth{cz + d}}
    \end{align*}
\end{boxlemma}

The proof uses results about the discriminant form, which we define in the next subsection. We do not prove the above result here, as it is significantly beyond the scope of this project, but we point the reader to the blueprint \cite[Lemma 6.39]{blueprint}.\todo{Update blueprint reference before submitting}

\subsection{The Discriminant Form}

The discriminant form is a weight $12$, level $1$ modular form. As was briefly alluded to earlier, it is a cusp form. It is defined in terms of the Eisenstein series $E_4$ and $E_6$.

\begin{boxdefinition}[The Discriminant Form]\label{Ch2:Def:DiscForm}
    The \textbf{discriminant form} $\Delta$ is defined by
    \begin{align}
        \Delta := \frac{E_4^3 - E_6^2}{1728}
        \label{Ch2:Eq:DiscForm_def}
    \end{align}
\end{boxdefinition}

The discriminant form has important positivity and non-vanishing properties that we will use repeatedly, either directly or indirectly, in the construction of the magic function. The discriminant form will often show up in denominators, making these properties essential to prove properties like holomorphicity. The key to these properties is the so-called product formula.

\begin{boxtheorem}[Product Formula for $\Delta$]\label{Ch2:Thm:Delta_Product_Formula}
    For all $z \in \Halfplane$, $\Delta(z)$ is expressible as the following infinite product:
    \begin{align}
        \Delta(z) = e^{2 \pi i z} \prod_{n=1}^{\infty} \parenth{1 - e^{2 \pi i n z}}^{24}
    \end{align}
\end{boxtheorem}

% ASK CHRIS ABOUT LEAN PROOF OF PRODUCT FORMULA!

A proof can be found in \cite[Chapter VII, §4, Theorem 6, p. 95]{SerreArith}. Birkbeck has \href{https://github.com/thefundamentaltheor3m/Sphere-Packing-Lean/blob/ba092be9cdebb1a9c170a22c234e71ca1842a173/SpherePacking/ModularForms/multipliable_lems.lean#L107}{shown formally} that the above product converges for all $z \in \Halfplane$.

As a remark, we mention that the theory of infinite products is not as well-developed in Lean as the theory of infinite sums. The definition of convergence of infinite products in \href{https://github.com/leanprover-community/mathlib4/blob/a98ecd2e7d46e2d29c4b572b1195c367b0106bf2/Mathlib/Topology/Algebra/InfiniteSum/Defs.lean#L92}{\mathlib} is designed to yield a strong notion of convergence of infinite sums involving invariance under rearrangements, and is stronger than the notion of pointwise convergence. We do not discuss the details here, but note that the condition is sufficiently strong for our purposes.

% Do we want to say anything at all about infinite products in Lean? I think it's a bad idea, because it's too much of a rabbit-hole (and will lead to too much overlap with formalising maths courseworks 1 and 2)

We now state the positivity and nonvanishing properties of $\Delta$ that we will use when constructing the magic function.

\begin{boxcorollary}
    The discriminant form has the following important properties.
    \begin{enumerate}
        \item For all $t > 0$, we have $\Delta\of{it} > 0$. That is, $\Delta$ is real and positive on the positive imaginary axis.
        \item For all $z \in \Halfplane$, $\Delta\of{z} \neq 0$. That is, $\Delta$ is nonvanishing on the upper half-plane.
    \end{enumerate}
\end{boxcorollary}

\subsection{The Theta Functions}
\label{Ch2:Subsec:ThetaFunctions}

In this subsection, we define and state some basic properties of the Theta functions $\Theta_2$, $\Theta_3$ and $\Theta_4$, the fourth powers of which define the corresponding $H$-functions. The $H$-functions will be important ingredients in the construction of the magic function.

\begin{boxdefinition}[The $\Theta$- and $H$-Functions]\label{Ch2:Def:Theta_H}
    Define $\Theta_2, \Theta_3, \Theta_4 : \Halfplane \to \C$ by
    \begin{align*}
        \Theta_2(z) &= \sum_{n \in \Z} e^{\pi i \parenth{n + \frac{1}{2}}^2 z} \\
        \Theta_3(z) &= \sum_{n \in \Z} e^{\pi i n^2 z} \\
        \Theta_4(z) &= \sum_{n \in \Z} \parenth{-1}^n e^{\pi i n^2 z}
    \end{align*}
    for all $z \in \Halfplane$. Define $H_2, H_3, H_4 : \Halfplane \to \C$ by
    \begin{align*}
        H_2 = \Theta_2^4 \qquad\qquad
        H_3 = \Theta_3^4 \qquad\qquad
        H_4 = \Theta_4^4
    \end{align*}
\end{boxdefinition}

It can be shown that the $H$-functions are modular forms of weight $2$ and level $2$.

Given the manner in which the $H$-functions are defined, it is tedious to compute their Fourier expansions explicitly. However, the purpose of computing the Fourier expansions of the Eisenstein Series was to determine that their Fourier coefficients grow polynomially. It turns out that in the case of the $H$-functions, we can do this without explicitly computing their Fourier series.

The Fourier coefficients of $H_3$ and $H_4$ grow polynomially because those of $\Theta_3$ and $\Theta_4$ grow polynomially\footnote{\Cref{Ch4:Prop:PolyGrowth_of_mul} explicitly establishes this fact.}: defining
\begin{align*}
    c_3(m) &=
    \begin{cases}
        1 \quad\quad\quad & \text{ if } m = n^2 \text{ for some } n \in \Z \\
        0 \quad\quad\quad & \text{ otherwise}
    \end{cases} \\
    c_4(m) &=
    \begin{cases}
        \parenth{-1}^n & \text{ if } m = n^2 \text{ for some } n \in \Z \\
        0 & \text{ otherwise}
    \end{cases}
\end{align*}
it is clear that $\abs{c_3(m)}, \abs{c_4(m)} \leq 1$ for all $m \in \Z$. The Fourier expansions of $\Theta_3$ and $\Theta_4$ are then given by
\begin{align*}
    \Theta_{3}\of{z} &= \sum_{m \in \Z} c_3(m) \, e^{i\pi m z} \\
    \Theta_{4}\of{z} &= \sum_{m \in \Z} c_4(m) \, e^{i\pi m z}
\end{align*}
The fact that the Fourier coefficients of $H_3$ and $H_4$ also grow polynomially can then be deduced by expressing $\Theta_3^4$ and $\Theta_4^4$ as iterated sums. This is tedious, and we do not do it here.

Unfortunately, due to the fractional term in the exponents of the summands in the definition of $\Theta_2$, it is not possible to use the same technique to show that its Fourier coefficients grow polynomially. Fortunately, we can still prove the result for $H_2$, because raising $\Theta_2$ to the fourth power gets rid of the fractional exponent. That is,
\begin{align}
    H_2 = \Theta_2^4
    &= \parenth{\sum_{n \in \Z} e^{\pi i \parenth{n + \frac{1}{2}}^2 z}}^4
    = \parenth{2\sum_{n =0}^{\infty} e^{\pi i \parenth{n + \frac{1}{2}}^2 z}}^4
    = \parenth{2\sum_{n =0}^{\infty} e^{\pi i \parenth{n^2 + n + \frac{1}{4}} z}}^4 \nonumber \\
    &= \parenth{2\sum_{n =0}^{\infty} e^{\pi i \parenth{n^2 + n} z} \, e^{\frac{\pi i z}{4}}}^4
    = \parenth{2e^{\frac{\pi i z}{4}}}^4  \parenth{\sum_{n =0}^{\infty} e^{\pi i \parenth{n^2 + n} z}}^4
    = 16 e^{\pi i z} \parenth{\sum_{n =0}^{\infty} e^{\pi i \parenth{n^2 + n} z}}^4 \label{Ch2:Eq:H2_qexpansion_explicit}
\end{align}
This can be explicitly computed as an iterated sum with coefficients that grow polynomially.

Finally, we mention some important relations that we will take advantage of when proving properties about the magic function. Some are given as slash actions of elements of $\SL{2, \Z}$, so we define some notation first.

\begin{boxnotation}
    Denote
    \begin{align*}
        S = \begin{bmatrix}
            0 & -1 \\ 1 & 0
        \end{bmatrix}
        \qquad \qquad
        T = \begin{bmatrix}
            1 & 1 \\ 0 & 1
        \end{bmatrix}
        \qquad \qquad
        I = \begin{bmatrix}
            1 & 0 \\ 0 & 1
        \end{bmatrix}
    \end{align*}
\end{boxnotation}

We now state important properties of the $H$-functions. These have been taken from \cite[\S 6]{blueprint}. The first version of of this section which was written by Viazovska herself, but it has subsequently been modified by Birkbeck and Lee, with contributions from Ma, to reflect the current state of the formalisation of the theory of modular forms, in \mathlib\ as well as the project repository and Birkbeck's own code.

\begin{boxproposition}\label{Ch2:Prop:H_Rels}
    The following slash action relations hold.
    \begin{align*}
        H_2 \mid_0 S &= -H_4
        &
        H_3 \mid_0 S &= -H_3
        &
        H_4 \mid_0 S &= -H_2
        \\
        H_2 \mid_0 T &= -H_2
        &
        H_3 \mid_0 T &= H_4
        &
        H_4 \mid_0 T &= H_3
    \end{align*}
    Furthermore, the $H$-functions are invariant under the weight $0$ slash action of $\Gamma(2)$. Finally, the $H$-functions are related to each other, $E_4$, $E_6$ and $\Delta$ in the following manner.
    \begin{align}
        0 &= H_2 - H_3 + H_4 \label{Ch2:Eq:H_Jacobi} \\
        E_4 &= \frac{1}{2} \parenth{H_2^2 + H_3^2 + H_4^2} \label{Ch2:Eq:E4_H} \\
        E_6 &= \frac{1}{2} \parenth{H_2 + H_3}\parenth{H_3 + H_4}\parenth{H_4 - H_2} \label{Ch2:Eq:E6_H} \\
        \Delta &= \frac{1}{256} \parenth{H_2 \, H_3 \, H_4}^2 \label{Ch2:Eq:Disc_H}
    \end{align}
    Further relations can be obtained by writing $H_3 = H_2 + H_4$ in \eqref{Ch2:Eq:E4_H} and \eqref{Ch2:Eq:E6_H}.
\end{boxproposition}

The proofs of these identities are significantly beyond the scope of this thesis.
