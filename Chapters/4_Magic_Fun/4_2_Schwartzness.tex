\section{Establishing the Schwartzness Property}

The magic function is a linear combination of $a$ and $b$, which are each defined as compositions of $a\rad$ and $b\rad$ with the norm-squared function. From \Cref{Ch3:Prop:Multidimensional_Schwartz_of_Schwartz}, we know that it is enough to establish that $a\rad$ and $b\rad$ are Schwartz to establish that $a$ and $b$ are Schwartz. In particular, this means the smoothness and decaying conditions need to be satisfied with respect to $\R$ inputs instead of $\R^8$ inputs, a substantial simplification. We can further simplify the problem by taking advantage of linearity.

We know that the Schwartz space is a $\C$-vector space, making it closed under addition. To show that $a\rad$ and $b\rad$ are Schwartz functions, we show that their constituent integrals $I_1, \ldots, I_6$ and $J_1, \ldots, J_6$ are Schwartz. We need to show both smoothness and rapid decay. Smoothness is fairly straightforward. Rapid decay, on the other hand, requires an additional ingredient.

It turns out that we can establish a general result that yields an upper-bound for functions of the form $\frac{f}{\Delta}$, where $\Delta$ is the discriminant form and there is a polynomial growth condition on the Fourier coefficients of $f$. We take advantage of the fact that the $\phi$- and $\psi$-functions can be expressed in this form (cf. \Cref{Ch4:Def:phis} and \Cref{Ch4:Prop:psi_as_div_disc}). The condition on their Fourier coefficients comes from the theory of modular forms.

We begin with the statement and proof of the general result \cite[Lemma 7.4]{blueprint}.\todo{UPDATE LEMMA NUMBER BEFORE SUBMITTING}

\begin{boxtheorem}\label{SP:PolyFourierCoeffBound}
    Let $f : \C \to \C$ be holomorphic. Denote by $c_f(n)$ its $n$th Fourier coefficient of $f$ with $c_f\of{n_0} \neq 0$, so that
    \begin{align*}
        f(z) = \sum_{n=n_0}^{\infty} c_f(n) \, e^{i \pi n z}
    \end{align*}
    If $c_f(n)$ has polynomial growth in $n$---that is, if there exists $k \in \N$ such that $c_f(n) = \BigO{n^k}$---then there exists a constant $C_f > 0$ such that for all $z \in \Halfplane$ with $\Im\of{z} \geq \frac{1}{2}$,
    \begin{align*}
        \abs{\frac{f(z)}{\Delta(z)}} \leq C_f \, e^{-\pi \parenth{n_0 - 2} \Im(z)}
    \end{align*}
\end{boxtheorem}
\begin{proof}
    
\end{proof}

As mentioned earlier, the strategy we will use to prove that the $\pm 1$-eigenfunctions are Schwartz is proving that their constituent integrals are Schwartz. The key to proving this is \Cref{SP:PolyFourierCoeffBound}: we establish that they satisfy the rapid decay condition by bounding their integrands above using \Cref{SP:PolyFourierCoeffBound}.

We now show that $a\rad$, the radial version of the $+1$-eigenfunction, is Schwartz.

\subsection{The $+1$-Eigenfunction}

We begin by proving that $I_1, \ldots, I_6$ are smooth. The key to proving this is the Leibniz Integral Rule\footnote{Also known as Leibniz's technique for `differentiating under the integral sign'}, which states that under mild conditions, the derivative with respect to one variable of the integral with respect to the other variable of a function of two variables is given by the integral of the corresponding partial derivative, which implies the analogous differentiability criterion.

\begin{boxlemma}
    For all $1 \leq j \leq 6$ and $k \in \N$, $I_j(r)$ is $k$ times differentiable.
\end{boxlemma}
\begin{proof}
    Fix $1 \leq j \leq 6$. We know, from \Cref{Ch4:Def:a}, that
    \begin{align*}
        I_j(r) = \int_{X_j} g_j\of{z} \, e^{\pi i r z} \, \diff{z}
    \end{align*}
    for intervals $X_j$ and holomorphic functions $g_j : \Halfplane \to \C$. The Leibniz Integral Rule then tells us that for all $k \in \N$, the $k$th derivative of $I_j$ at some $r \in \R$ is given by
    \begin{align}
        \int_{X_j} g_j\of{z} \, \parenth{\pi i z}^{k} \, e^{\pi i r z} \, \diff{z}
        \label{Ch4:Eq:Ij_deriv}
    \end{align}
    In particular, $I_j$ is smooth (in $r$) for all $j$.
\end{proof}

We are now ready to prove that $I_1, \ldots, I_6$ and their derivatives decay rapidly enough. The strategy to do this will be to bound the $I_j$ (and their by derivatives) rapidly enough decaying functions. For conciseness, we only prove the result for the zeroth derivatives of $I_1$, $I_2$ and $I_6$ and only reduce the $n$th derivative case to the zeroth derivative case for $I_1$. The others are proved quite similarly.

As a first step towards bounding the $I_j$, we need to show that we can apply \Cref{SP:PolyFourierCoeffBound} to obtain upper-bounds for $\phi_0$, $\phi_{-2}$ and $\phi_{-4}$.

\begin{boxlemma}\label{Ch4:Lemma:PolyFourierCoeffBound_Apply}
    There exist real numbers $C_0, C_{-2}, C_{-4} > 0$ such that
    \begin{align*}
        \abs{\phi_0\of{z}} &\leq C_{0} e^{-2\pi\Im(z)} \\
        \abs{\phi_{-2}\of{z}} &\leq C_{-2} \\
        \abs{\phi_{-4}\of{z}} &\leq C_{-4} e^{2\pi\Im(z)}
    \end{align*}
    for all $z \in \Halfplane$ with $\Im(z) > \frac{1}{2}$.
\end{boxlemma}
\begin{proof}
    Fix $z \in \Halfplane$ and assume that $\Im(z) > \frac{1}{2}$. The following is a consequence of a result proved by Ramanujan \todo{find citation for original paper} in the theory of (quasi-)modular forms:
    \begin{align}
        E_2 E_4 - E_6 &= 720 \sum_{n=1}^{\infty} n \, \sigma_3(n) \, e^{2\pi i nz}
        \label{Ch4:Eq:E2E4_sub_E6_qexpansion}
    \end{align}
    with the sum converging absolutely. Furthermore, it is easily seen that $\sigma_3(n) = \BigO{n^4}$, since $\sigma_3(n)$ is the sum of at most $n$ elements that are each at most $n^3$. Then, combining \eqref{Ch4:Eq:E2E4_sub_E6_qexpansion} and \eqref{Ch2:Eq:E4_qexpansion}, one can show that $E_4^2$, $E_4\parenth{E_2 E_4 - E_6}$, and $\parenth{E_2 E_4 - E_6}$ all have Fourier expansions in which each Fourier coefficient is $\BigO{n^{10}}$. For instance, we know
    \begin{align}
        \parenth{E_2 E_4 - E_6}^2 = 720^2 \sum_{m=1}^{\infty} \sum_{n=1}^{\infty} n \, \sigma_3(n) \, e^{2\pi i z \parenth{m + n}}
        \label{Ch4:Eq:E2E4_sub_E6_sq_qexpansion}
    \end{align}
    The Fourier coefficient corresponding to a term of the form $e^{2 \pi i z \parenth{m + n}}$ is a sum of at most $m + n$ terms that are each bounded above by $\parenth{\parenth{m + n}\sigma_3\of{m + n}}^2$, which is $\BigO{\parenth{m + n}^{10}}$. Similar computations can be performed for $E_4^2$ and $E_4\parenth{E_2 E_4 - E_6}$, proving that the polynomial growth assumption of \Cref{SP:PolyFourierCoeffBound} is satisfied by $\phi_{-4}$, $\phi_{-2}$ and $\phi_0$.

    We also note from \eqref{Ch4:Eq:E2E4_sub_E6_sq_qexpansion} that the smallest nonzero Fourier coefficient corresponds to $m + n = 2$. Reconciling this with the fact that in the statement of \Cref{SP:PolyFourierCoeffBound}, we express the Fourier series in terms of powers of $e^{\pi i z}$ and not $e^{2 \pi i z}$, we conclude that the right choice of $n_0$ to bound $\phi_0$ is $4$. Similar computations show that the right value for $\phi_{-2}$ is $2$ and that for $\phi_{-4}$ is $0$, which allows us to apply \Cref{SP:PolyFourierCoeffBound} and conclude that there exist $C_0, C_{-2}, C_{-4} > 0$ such that
    \begin{align*}
        \abs{\phi_0\of{z}} &\leq C_{0} e^{-2\pi\Im(z)} \\
        \abs{\phi_{-2}\of{z}} &\leq C_{-2} \\
        \abs{\phi_{-4}\of{z}} &\leq C_{-4} e^{2\pi\Im(z)}
    \end{align*}
    as required.
\end{proof}

We can now bound $I_1$, $I_3$ and $I_5$.

\begin{boxlemma}\label{Ch4:Lemma:Bound_I1_I3_I5}
    There exists a positive real number $C_0$ such that for all $r \in \R$,
    \begin{align*}
        \abs{I_1(r)}, \abs{I_3(r)}, \abs{I_5(r)} &\leq \int_{1}^{\infty} C_0 \, e^{-2\pi s} \, e^{-\pi r/s} \, \diff{s}
    \end{align*}
\end{boxlemma}
\begin{proof}
    For conciseness, we only bound $\abs{I_1}$ explicitly. Parametrise $z = -1 + it$ in \eqref{Ch4:Eq:I_1_def}. Then, for all $r \in \R$, we can write
    \begin{align*}
        I_1(r) &= -i \int_{0}^{1}
            \phi_0\of{\frac{-1}{it}} \,
            t^2 \,
            e^{-\pi i r} \,
            e^{\pi r t}\,
            \diff{t}
    \end{align*}
    Writing $s = \frac{1}{t}$ and simplifying, we get that
    \begin{align*}
        I_1(r) &= -i \int_{1}^{\infty}
            \phi_0\of{is} \,
            s^{-4} \,
            e^{-\pi i r} \,
            e^{-\pi r / s}\,
            \diff{t}
    \end{align*}
    Applying the triangle inequality, multiplicativity and monotonicity, we get
    \begin{align*}
        \abs{I_1(r)} &\leq \int_{1}^{\infty} \abs{
            \phi_0\of{is} \,
            s^{-4} \,
            e^{-\pi i r} \,
            e^{-\pi r / s}\,
            } \diff{t}
        \leq \int_{1}^{\infty}
            \abs{\phi_0\of{is}} \,
            e^{-\pi i r/s}
    \end{align*}
    Since $s > \frac{1}{2}$ inside the integral, we know from \Cref{Ch4:Lemma:PolyFourierCoeffBound_Apply} that $\exists C_0 > 0$ such that
    \begin{align*}
        \abs{I_1(r)} \leq \int_{1}^{\infty} C_0 \, e^{-2\pi s} \, e^{-\pi r/s} \, \diff{s}
    \end{align*}
    as required. The bounds on $\abs{I_3}$ and $\abs{I_5}$ are computed similarly.
\end{proof}

Now that we have bounds on the integrals with bounded vertical contours, we compute bounds on the integrals with bounded horizontal contours.

\begin{boxlemma}\label{Ch4:Lemma:Bound_I2_I4}
    There exists a positive real number $C_1$ such that for all $r \in \R$,
    \begin{align*}
        \abs{I_2(r)}, \abs{I_4(r)} &\leq C_1 \, e^{-\pi r}
    \end{align*}
\end{boxlemma}
\begin{proof}
    For conciseness, we only bound $\abs{I_2}$ explicitly. Parametrise $z = -1 + t + i$ in \eqref{Ch4:Eq:I_2_def}. Then, for all $r \in \R$, we can write
    \begin{align*}
        I_2(r) = \int_{0}^{1}
            \phi_0\of{\frac{-1}{t + i}} \,
            \parenth{t + i}^2 \,
            e^{-\pi i r} \,
            e^{\pi i r t} \,
            e^{-\pi r} \,
            \diff{t}
    \end{align*}
    Applying the triangle inequality, multiplicativity and monotonicity, we get
    \begin{align*}
        \abs{I_2(r)} &\leq \int_{0}^{1} \abs{\phi_0\of{\frac{-1}{t + i}}} \cdot 2e^{-\pi r} \, dt
    \end{align*}
    It is therefore enough to show that \Cref{SP:PolyFourierCoeffBound} applies. To that end, we manipulate the expression inside $\phi_0$:
    \begin{align*}
        \frac{-1}{t + i} = \frac{t}{t^2 + 1} + \frac{1}{t^2 + 1}i
    \end{align*}
    From this, we can deduce that its imaginary part is greater than $\frac{1}{2}$ for all $t \in \parenth{0, 1}$. Then, \Cref{SP:PolyFourierCoeffBound} tells us that $\exists C_0 > 0$ such that
    \begin{align*}
        \abs{\phi_0\of{\frac{-1}{t + i}}}
        \leq C_0 \, e^{-2 \pi \cdot \frac{1}{t^2 + 1}}
        \leq C_0 \, e^{-2\pi \cdot \frac{1}{2}} = C_0 \, e^{-\pi}
    \end{align*}
    Then, taking $C_1 := 2 \, C_0 \, e^{-\pi}$ and applying monotonicity,
    \begin{align*}
        \abs{I_2(r)} \leq \int_{0}^{1} C_1 \, e^{-\pi r} \, \diff{t} = C_1 \, e^{-\pi r}
    \end{align*}
    as required. The bound on $\abs{I_4}$ is computed similarly.
\end{proof}

Finally, we bound the integral with the unbounded vertical contour.

\begin{boxlemma}\label{Ch4:Lemma:Bound_I6}
    There exists a positive real number $C_2$ such that for all $r \in \R$,
    \begin{align*}
        \abs{I_6(r)} &\leq C_2 \, \frac{e^{-\pi \parenth{r + 2}}}{r + 2}
    \end{align*}
\end{boxlemma}
\begin{proof}
    Parametrise $z = it$ in \eqref{Ch4:Eq:I_6_def}. Then, for all $r \in \R$, we can write
    \begin{align*}
        I_6(r) &= -2i \int_{1}^{\infty} \phi_0\of{it} \, e^{- \pi r t} \, \diff{t}
    \end{align*}
    Applying the triangle inequality, multiplicativity and monotonicity, we get
    \begin{align*}
        \abs{I_6(r)} \leq 2\int_{1}^{\infty}
            \abs{\phi_0\of{it}} \,
            e^{-\pi r t} \,
            \diff{t}
    \end{align*}
    Let $C_0$ be as in \Cref{SP:PolyFourierCoeffBound} and define $C_1 := 2 C_0$. Then,
    \begin{align*}
        \abs{I_6(r)} \leq 2\int_{1}^{\infty}
            C_0 \,
            e^{-2 \pi t} \,
            e^{-\pi r t} \,
            \diff{t}
        = C_1 \int_{1}^{\infty} e^{-\parenth{2\pi + \pi r}t} \, \diff{t}
        = C_1 \frac{e^{-\pi\parenth{r + 2}}}{\pi\parenth{r + 2}}
    \end{align*}
    Defining $C_2 := \frac{C_1}{\pi}$ then yields the desired result.
\end{proof}

The formal proofs of the bounds in \Cref{Ch4:Lemma:Bound_I1_I3_I5,Ch4:Lemma:Bound_I2_I4,Ch4:Lemma:Bound_I6} are complete up to proofs that they are bounded by integrable functions, which is necessary to apply monotonicity of the integral due to the definition of the integral in \mathlib. The evaluation of the final integral in \Cref{Ch4:Lemma:Bound_I6} is also currently a \sorry.\todo{Update}

We now demonstrate how bounds on higher derivatives of $I_1$ can be reduced to the case we saw in \Cref{Ch4:Lemma:Bound_I1_I3_I5}.

\begin{boxlemma}
    For all $k \in \N$, there exists a positive real number $C_0^{(k)}$ such that for all $r \in \R$,
    \begin{align*}
        \abs{I_1^{(k)}\of{r}} \leq \int_{1}^{\infty} C_0^{(k)} \, e^{-2\pi s} \, e^{- \pi r / s} \, \diff{s}
    \end{align*}
\end{boxlemma}
\begin{proof}
    Fix $k \in \N$ and $r \in \R$. As we showed in \eqref{Ch4:Eq:Ij_deriv}, the $k$th derivative of $I_1$ at $r$ can be expressed as
    \begin{align*}
        I_1^{(k)}\of{r} = \int_{-1}^{-1+i} \phi_0\of{\frac{-1}{z+1}} \, \parenth{z+1}^2 \, \parenth{\pi i z}^k \, e^{\pi i r z} \, \diff{z}
    \end{align*}
    Parametrise $z = -1 + it$. Then,
    \begin{align*}
        I_1^{(k)}\of{r} = -i \int_{0}^{1} \phi_0\of{\frac{-1}{it}} \, t^2 \, \parenth{-\pi i - \pi t}^k \, e^{-\pi i r} \, e^{\pi r t} \, \diff{z}
    \end{align*}
    Writing $s = \frac{1}{t}$ and simplifying, we get that
    \begin{align*}
        I_1^{(k)}\of{r} &= -i \int_{1}^{\infty}
            \phi_0\of{is} \,
            s^{-4} \,
            \parenth{-\pi i - \frac{\pi}{s}}^k \,
            e^{-\pi i r} \,
            e^{-\pi r / s}\,
            \diff{t}
    \end{align*}
    Observe that for all $s \in \Ico{1, \infty}$,
    \begin{align*}
        \abs{-\pi i - \frac{\pi}{s}} = \pi \abs{i + \frac{1}{s}} \leq \pi \sqrt{2}
    \end{align*}
    Combining this with the bound computed in the proof of \Cref{Ch4:Lemma:Bound_I1_I3_I5}, we know $\exists C_0 > 0$ such that
    \begin{align*}
        \abs{I_1^{(k)}\of{r}} \leq \int_{1}^{\infty} C_0 \parenth{\pi \sqrt{2}}^{k} \, e^{-2\pi s} \, e^{- \pi r / s} \, \diff{s}
    \end{align*}
    Defining $C_0^{(k)} := C_0 \parenth{\pi \sqrt{2}}^{k}$ then gives the desired result.
\end{proof}

Analogous results can be proved for the other $I_j$, with any arising constants subsumed into the constants defined in \Cref{Ch4:Lemma:Bound_I1_I3_I5,Ch4:Lemma:Bound_I2_I4,Ch4:Lemma:Bound_I6}. Therefore, if we can show that the functions (without the constants) on the right-hand sides of \Cref{Ch4:Lemma:Bound_I1_I3_I5,Ch4:Lemma:Bound_I2_I4,Ch4:Lemma:Bound_I6} decay faster than any inverse power of $r$, we will have the result.

This fact is obvious for $\abs{I_2}$, $\abs{I_4}$ and $\abs{I_6}$, but less so for $\abs{I_1}$, $\abs{I_3}$ and $\abs{I_5}$. In these cases, the result is actually a consequence of a deeper result involving modified Bessel functions of the second kind.

% Existing proof is wrong... find workaround (or a good reference).

An important comment we make here is that the fact that the functions that (up to constants) bound the $I_j$ and their derivatives decay rapidly also establishes that the integrals $I_j$ converge absolutely. In particular, their integrands are integrable, which will be a stepping stone to applying results like Fubini's Theorem later on.

We now show that similar results hold for the $-1$-eigenfunction.

\subsection{The $-1$-Eigenfunction}

A theme we will see repeatedly is that there are analogues for the $-1$-eigenfunction of many results on the $+1$-eigenfunction. The proof of the Schwartzness property is no exception to this rule.

Given that the contours used are similar, the main step in which the proof that $b\rad$ is Schwartz differs from the proof that $a\rad$ is Schwartz is the step in which we show that \Cref{SP:PolyFourierCoeffBound} is applicable.