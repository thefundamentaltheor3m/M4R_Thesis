\section{Establishing the Schwartzness Property}

We know that the Schwartz space is a $\C$-vector space, making it closed under addition. To show that $a$ and $b$ are Schwartz functions, we show that their constituent integrals $I_1, \ldots, I_6$ and $J_1, \ldots, J_6$ are Schwartz. We need to show both smoothness and rapid decay. Smoothness is fairly straightforward. Rapid decay, on the other hand, requires an additional ingredient.

It turns out that we can establish a general result that yields an upper-bound for functions of the form $\frac{f}{\Delta}$, where $\Delta$ is the discriminant form and there is a polynomial growth condition on the Fourier coefficients of $f$. We take advantage of the fact that the $\phi$- and $\psi$-functions can be expressed in this form (cf. \Cref{Ch4:Def:phis} and \Cref{Ch4:Prop:psi_as_div_disc}). The condition on their Fourier coefficients comes from the theory of modular forms.

We begin with the statement and proof of the general result \cite[Lemma 7.4]{blueprint}.\todo{UPDATE LEMMA NUMBER BEFORE SUBMITTING}

\begin{boxtheorem}\label{SP:PolyFourierCoeffBound}
    Let $f : \C \to \C$ be holomorphic. Denote by $c_f(n)$ its $n$th Fourier coefficient of $f$ with $c_f\of{n_0} \neq 0$, so that
    \begin{align*}
        f(z) = \sum_{n=n_0}^{\infty} c_f(n) \, e^{i \pi n z}
    \end{align*}
    If $c_f(n)$ has polynomial growth in $n$---that is, if there exists $k \in \N$ such that $c_f(n) = \BigO{n^k}$---then there exists a constant $C_f > 0$ such that for all $z \in \Halfplane$ with $\Im\of{z} \geq \frac{1}{2}$,
    \begin{align*}
        \abs{\frac{f(z)}{\Delta(z)}} \leq C_f \, e^{-\pi \parenth{n_0 - 2} \Im(z)}
    \end{align*}
\end{boxtheorem}


\subsection{The $+1$-Eigenfunction}

\subsection{The $-1$-Eigenfunction}