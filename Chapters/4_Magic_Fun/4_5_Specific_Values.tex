\section{Another Representation of the Eigenfunctions}

We have already established numerous properties of $a$ and $b$: we have shown they are Schwartz, that they are indeed $\pm1$-eigenfunctions of the Fourier transform, and that they have double zeroes at all points of $\Lambda_8$ that have norm $> \sqrt{2}$. However, given that our objective is to find a linear combination of $a$ and $b$ satisfying \ref{CE1}-\ref{CE3}, we need information about the behaviour of $a\rad$ and $b\rad$ on specific points in $\Ico{0, \infty}$.

Proving \ref{CE1} is not difficult. Proving \ref{CE2} is quite non-trivial, but the key ingredient is the representation of $a\rad$ as $d$ and $b\rad$ as $c$ for $r > 2$. Since $a$ and $b$ are $\pm 1$-Fourier eigenfunctions, we only need to prove that the linear combination with one sign flipped satisfies \ref{CE3}. This, however, requires insight about the behaviour of $a\rad$ and $b\rad$ on all of $\Ioc{0, \infty}$. Since the defining integrals $I_j$ and $J_j$ are difficult to evaluate directly, Viazovska's approach was to analytically continue $d$ and $c$.

\subsection{The $+1$-Eigenfunction}

We begin by defining the following integral.

\begin{boxdefinition}
    Define $\tilde{d} : \Ioc{0, \infty} \to \R$ by
    \begin{align*}
        \tilde{d}(r) := \int_{0}^{\infty} \parenth{t^2 \phi_0\of{\frac{i}{t}} - \frac{36}{\pi^2}e^{2\pi t} + \frac{8640}{\pi} - 18144} e^{-\pi r t} \, \diff{t}
    \end{align*}
\end{boxdefinition}

