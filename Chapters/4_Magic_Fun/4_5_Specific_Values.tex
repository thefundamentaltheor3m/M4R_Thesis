\section{Another Representation of the Eigenfunctions}
\label{Ch4:Sec:Analytic_Cont}

We have already established numerous properties of $a$ and $b$: we have shown they are Schwartz, that they are indeed $\pm1$-eigenfunctions of the Fourier transform, and that they have double zeroes at all points of $\Lambda_8$ that have norm $> \sqrt{2}$. However, given that our objective is to find a linear combination of $a$ and $b$ satisfying \ref{CE1}-\ref{CE3}, we need information about the behaviour of $a\rad$ and $b\rad$ on specific points in $\Ico{0, \infty}$. We will show that \ref{CE2} is a consequence of the alternate representation constructed in \Cref{Ch4:Sec:Double_Zeroes}. However, $d$ and $c$ are only defined on $\parenth{2, \infty}$, so they do not provide us with information on what happens closer to $0$, which we need for \ref{CE3}, This will also help us prove \ref{CE1} by showing us that $a(0) \neq 0$, and help us show that the Cohn-Elkies bound we get in the end does indeed give us the density of the $\Lambda_8$.

\subsection{The $+1$-Eigenfunction}

We begin by defining the following integral.

\begin{boxdefinition}
    Define $\tilde{d} : \Ico{0, \infty} \to \R$ by
    \begin{align*}
        \tilde{d}(r) := \int_{0}^{\infty} \parenth{t^2 \phi_0\of{\frac{i}{t}} - \frac{36}{\pi^2}e^{2\pi t} + \frac{8640}{\pi} - \frac{18144}{\pi^2}} e^{-\pi r t} \, \diff{t}
    \end{align*}
\end{boxdefinition}

Observe that for $r > 2$
\begin{align*}
    \int_{0}^{\infty} \parenth{\frac{36}{\pi^2}e^{2\pi t} - \frac{8640}{\pi} + \frac{18144}{\pi^2}} e^{-\pi r t} \, \diff{t}
    = \frac{36}{\pi^3\parenth{r - 2}} - \frac{8640}{\pi^3 r^2} + \frac{18144}{\pi^3 r}
\end{align*}
In particular, the integral in $\tilde{d}$ converges. Furthermore, we have
\begin{align*}
    \tilde{d}(r) = d(r) + \frac{36}{\pi^3\parenth{r - 2}} - \frac{8640}{\pi^3 r^2} + \frac{18144}{\pi^3 r}
\end{align*}
for all $r > 2$. So $\tilde{d}$ is a continuation of $d$. However, it is not immediately clear that $\tilde{d}$ is analytic.