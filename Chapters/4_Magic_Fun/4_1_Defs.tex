\section{Defining Viazovska's Fourier Eigenfunctions}

The $\pm$-eigenfunctions of $g$---and, by extension, $g$ itself---are defined in terms of modular and quasimodular forms (recall the definitions of these terms from \Cref{Ch2:Sec:ModForms}). Specifically, the $+1$-eigenfunction $a$ is defined in terms of the so-called $\phi$- and $\psi$-functions, which are in turn defined in terms of the Eisenstein series (cf. \sorry) and the discriminant form (cf. \sorry), while the $-1$-eigenfunction $b$ is defined in terms of the Jacobi Theta functions (cf. \sorry)\todo{Define in Chapter 2 and cross-ref here}.

We begin by defining the $+1$-eigenfunction $a$.

\subsection{The $+1$-Eigenfunction}

We begin by defining the $\phi$-functions.

\begin{boxdefinition}[The $\phi$-Functions]\label{Ch4:Def:phis}
    Define the functions $\phi_0, \phi_{-2}, \phi_{-4} : \Halfplane \to \C$ by
    \begin{align}
        \phi_{-4} &:= \frac{E_4^2}{\Delta}
            \label{Ch4:Eq:phi_-4_def} \\
        \phi_{-2} &:= \frac{E_4\parenth{E_2 E_4 - E_6}}{\Delta}
            \label{Ch4:Eq:phi_-2_def} \\
        \phi_{0} &:= \frac{\parenth{E_2 E_4 - E_6}^2}{\Delta}
            \label{Ch4:Eq:phi_0_def}
    \end{align}
\end{boxdefinition}

These functions admit important transformation properties that are necessary to prove that the $+1$-eigenfunction is made up of integrals of holomorphic functions. This fact will in turn allow us to apply the Cauchy-Goursat Theorem (and variants thereof) that will allow us to shift contours of integration.

\begin{boxlemma}
    For all $z \in \Halfplane$,
    \begin{align}
        \phi_0\of{z + 1}
        &= \phi_0\of{z}
        \label{Ch4:Eq:phi_0_add_one} \\
        \phi_0\of{\frac{-1}{z}}
        &= \phi_0\of{z}
        - \frac{12 i}{\pi} \cdot \frac{1}{z} \cdot \phi_{-2}\of{z}
        - \frac{36}{\pi^2} \cdot \frac{1}{z^2} \cdot \phi_{-4}\of{z}
        \label{Ch4:Eq:phi_0_neg_inv}
    \end{align}
\end{boxlemma}

The proofs of these involve transformations of $E_2$, $E_4$, $E_6$, and $\Delta$, and can be found in \cite{blueprint}. We now define the $+1$-eigenfunction $a$.

\begin{boxdefinition}[Viazovska's $+1$-Fourier Eigenfunction]\label{Ch4:Def:a}
    Define $a\rad : \R \to \C$ by
    \begin{align}
        a\rad\of{r} := I_1(r) + I_2(r) + I_3(r) + I_4(r) + I_5(r) + I_6(r)
            \label{Ch4:Eq:a_rad_def}
    \end{align}
    where, for all $r \in \R$,
    \begin{align}
        I_1(r) &:= \int_{-1}^{-1 + i} \phi_0\of{\frac{-1}{z+1}} \,
                                 \parenth{z + 1}^2 \,
                                 e^{\pi i r z} \,
                                 \diff{z}
            \label{Ch4:Eq:I_1_def} \\
        I_2(r) &:= \int_{-1 + i}^{i} \phi_0\of{\frac{-1}{z+1}} \,
                                 \parenth{z + 1}^2 \,
                                 e^{\pi i r z} \,
                                 \diff{z}
            \label{Ch4:Eq:I_2_def} \\
        I_3(r) &:= \int_{1}^{1 + i} \phi_0\of{\frac{-1}{z - 1}} \,
                                \parenth{z - 1}^2 \,
                                e^{\pi i r z} \,
                                \diff{z}
            \label{Ch4:Eq:I_3_def} \\
        I_4(r) &:= \int_{1 + i}^{i} \phi_0\of{\frac{-1}{z - 1}} \,
                                \parenth{z - 1}^2 \,
                                e^{\pi i r z} \,
                                \diff{z}
            \label{Ch4:Eq:I_4_def} \\
        I_5(r) &:= -2 \int_{0}^{i} \phi_0\of{\frac{-1}{z}} \,
                                z^2 \,
                                e^{\pi i r z} \,
                                \diff{z}
            \label{Ch4:Eq:I_5_def} \\
        I_6(r) &:= 2 \int_{i}^{i \infty} \phi_0\of{z} \,
                                e^{\pi i r z} \,
                                \diff{z}
            \label{Ch4:Eq:I_6_def}
    \end{align}
    Define the $+1$-Fourier eigenfunction $a : \R^8 \to \C$ by
    \begin{align}
        a(x) := a\rad\of{\norm{x}^2}
            \label{Ch4:Eq:a_def}
    \end{align}
\end{boxdefinition}

It is immediate from \eqref{Ch4:Eq:a_def} that $a$ is radial. All of its properties are determined by its radial part $a\rad$. There are similar definitions in Lean.

There is an important remark that must be made about the definitions in \eqref{Ch4:Eq:I_1_def}-\eqref{Ch4:Eq:I_6_def}: in the original paper \cite{Viazovska8}, the integrals $I_1$ and $I_2$ are combined, as are $I_3$ and $I_4$, and expressed in the following manner:
\begin{align*}
    I_1(r) + I_2(r) &= \int_{-1}^{i} \phi_0\of{\frac{-1}{z+1}} \, \parenth{z + 1}^2 \, e^{\pi i r z} \, \diff{z} \\
    I_3(r) + I_4(r) &= \int_{1}^{i} \phi_0\of{\frac{-1}{z - 1}} \, \parenth{z - 1}^2 \, e^{\pi i r z} \, \diff{z}
\end{align*}
with the contours not specified. The most `classical' choice would be quarter-circular contours, though the same results can be achieved working with straight and rectangular contours.

\begin{figure}[ht]
    \centering
    % First subfigure: quarter-circular contour
    \begin{subfigure}{0.3\textwidth}
        \centering
        \begin{tikzpicture}[scale=1.75]
            % Axes
            \draw[->] (-1.2,0) -- (1.2,0) node[right] {$\Re$};
            \draw[->] (0,-1.2) -- (0,1.2) node[above] {$\Im$};

            % Quarter-circle from -1 to i
            \draw[thick, domain=180:135, ->] plot ({cos(\x)}, {sin(\x)}) node[above left] {$I_1 + I_2$};
            \draw[thick, domain=135:90, -] plot ({cos(\x)}, {sin(\x)}) ;

            % Points of interest
            \labelledpoint{-1}{0}{0}{-0.8}{$-1$}
            \labelledpoint{0}{1}{0.25}{-0.4}{$i$}
        \end{tikzpicture}
        \label{Ch4:subfig:a_circ_contour}
        \caption{Quarter-Circular Contour}
    \end{subfigure}
    \hfill
    % Second subfigure: straight line
    \begin{subfigure}{0.3\textwidth}
        \centering
        \begin{tikzpicture}[scale=1.75]
            % Axes
            \draw[->] (-1.2,0) -- (1.2,0) node[right] {$\Re$};
            \draw[->] (0,-1.2) -- (0,1.2) node[above] {$\Im$};

            % Straight line from -1 to 1
            \draw[thick, ->] (-1,0) -- (-0.5,0.5) node[above left] {$I_1 + I_2$};
            \draw[thick, -] (-0.5,0.5) -- (0,1);

            % Points of interest
            \labelledpoint{-1}{0}{0}{-0.8}{$-1$}
            \labelledpoint{0}{1}{0.25}{-0.4}{$i$}
        \end{tikzpicture}
        \label{Ch4:subfig:a_lin_contour}
        \caption{Straight Line Contour}
    \end{subfigure}
    \hfill
    % Third subfigure: rectangular contour
    \begin{subfigure}{0.3\textwidth}
        \centering
        \begin{tikzpicture}[scale=1.75]
            % Axes
            \draw[->] (-1.2,0) -- (1.2,0) node[right] {$\Re$};
            \draw[->] (0,-1.2) -- (0,1.2) node[above] {$\Im$};

            % Rectangular contour
            \draw[thick, ->] (-1,0) -- (-1,0.5) node[left] {$I_1$};
            \draw[thick, -] (-1,0.5) -- (-1,1);
            \draw[thick, ->] (-1,1) -- (-0.5,1) node[above] {$I_2$};m
            \draw[thick, -] (-0.5,1) -- (0,1);

            % Points of interest
            \labelledpoint{-1}{0}{0}{-0.8}{$-1$}
            \labelledpoint{0}{1}{0.25}{-0.4}{$i$}
            \labelledpoint{-1}{1}{-0.7}{-0.2}{$-1 + i$}
        \end{tikzpicture}
        \label{Ch4:subfig:a_rect_contour}
        \caption{Rectangular Contour}
    \end{subfigure}
    \caption{\centering Different contours along which we can integrate the integrand of $I_1$ and $I_2$ to get an integral equal to $I_1 + I_2$}
    \label{Ch4:fig:a_contours}
\end{figure}

The reason the choice of contours does not matter is that in the integrands of $I_1, \ldots, I_5$, we multiply terms of the form $\phi_0\of{\frac{-1}{z}}$ by $z^2$. If we apply \eqref{Ch4:Eq:phi_0_neg_inv} and multiply through, it is clear that we are removing any singularities introduced by $\frac{1}{z^2}$ and $\frac{1}{z}$ factors. We can then use the fact that $\Delta\of{z} \neq 0$ for all $z \in \Halfplane$ to conclude that the integrands are holomorphic up to these removable singularities.

The choice of rectangular contours (as in \Cref{Ch4:subfig:a_rect_contour}) as opposed to quarter-circles or straight lines for $I_1 + I_2$ and $I_3 + I_4$ is motivated by the versions of the Cauchy-Goursat Theorem that have been formalised in Lean.

% Insert a visual here for I_1, and say I_2 is analogous

We are now ready to define the $-1$-eigenfunction $b$.

\subsection{The $-1$-Eigenfunction}

Recall the definitions of the $H$-functions defined in \sorry\ to be the fourth powers of the thetanullwerte. We begin by defining the $h$ function, in terms of which we will define the $\psi$-functions.

\begin{boxdefinition}[The $h$-Function]\label{Ch4:Def:h}
    Define the function $h : \Halfplane \to \C$ by
    \begin{align}
        h\of{z} := 128 \frac{H_3(z) + H_4(z)}{H_2(z)^2} \label{Ch4:Eq:h_def}
    \end{align}
    where $H_2, H_3, H_4$ are as defined in \sorry.
\end{boxdefinition}

In \cite{Viazovska8}, the $\psi$-functions are defined in terms of the $h$-function via slash actions. We express them in terms of $h$ more directly.

\begin{boxdefinition}[The $\psi$-Functions]\label{Ch4:Def:psis}
    Define the functions $\psi_I, \psi_S, \psi_T : \Halfplane \to \C$ by
    % \begin{align} % What nonsense have I written... gotta fix this! XD
    %     \psi_I\of{z} &:= h\of{z}^3
    %         \label{Ch4:Eq:psi_I_def} \\
    %     \psi_S\of{z} &:= -h\of{z}^3
    %         \label{Ch4:Eq:psi_S_def} \\
    %     \psi_T\of{z} &:= h\of{z}^3 - h\of{z}^3
    %         \label{Ch4:Eq:psi_T_def}
    % \end{align}
    where $h$ is as defined in \Cref{Ch4:Def:h}.
\end{boxdefinition}

\begin{boxproposition}[The $\psi$-Fucntions]\label{Ch4:Prop:psi_as_div_disc}
    We can express $\psi_I, \psi_S, \psi_T$ in the following manner:
    \begin{align}
        \psi_I &= \frac{H_4^3\parenth{2 H_4^2 + 5 H_4 H_2 + 5 H_2^2}}{2 \Delta}
            \label{Ch4:Eq:psi_I_def} \\
        \psi_S &= \frac{- H_2^3 \parenth{2 H_2^3 + 5 H_2 H_4 + 5 H_4^2}}{2 \Delta}
            \label{Ch4:Eq:psi_S_def} \\
        \psi_T &= \psi_I - \psi_S
            \label{Ch4:Eq:psi_T_def}
    \end{align}
    where $\Delta$ is the discriminant form.
\end{boxproposition}