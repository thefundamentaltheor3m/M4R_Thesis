\section{A Glance Ahead}

Despite still being fairly far from complete, this formalisation effort has all the makings of a paradigm-shifter, both for the formalisation community and the mathematics community at large. Already, the fact that \mathlib\ is at a stage where it is possible to envision a formalisation of such recent and revolutionary work as Viazovska's says volumes about the progress in formalisation over the last decade. Moreover, it is a testament to the power of open-source collaboration: every day, strides are being made by countless contributors worldwide that enhance the capabilities of Lean and the vastness of \mathlib, making for a more efficient and enjoyable formalisation experience. It is to harness the power of this community and to mobilise enthusiasts and experts to contribute to this project that the collaborators, represented by Viazovska herself, have decided to make the repository public on 13 June, 2025, at Cambridge.

Beyond the specifics of this project, the goals as well as the accomplishments thus far have a wide range of applications. For example, the author's bridge between dimensions involving Schwartz-like functions, seen in \Cref{Ch5:Subsec:Schwartz_Bridge}, and the fact that the Schwartz-like behaves nicely with linear combinations, as used in \Cref{Ch4:Sec:Schwartzness} to prove that $a\rad$ and $b\rad$ are Schwartz-like, suggests that there is a deeper theory waiting to be explored, an exciting prospect so soon after Viazovska added a revolutionary new perspective to the theory of Schwartz functions. On the formal front, the development of \lstinline|norm_numI| at a time when there is much discussion on improving automation in a variety of algebraic settings could lead to further improvements being made to existing tactics, potentially leading to the development of polymorphic automation, where normalisations and simplifications understand and depend on the setting. Furthermore, just as this project catalysed the development of \lstinline|norm_numI|, so too may it catalyse formalisations of results like the \CGT\ and potentially even the \JCT, so that future formalisers can sidesteps the vexation faced by the author when dealing with contour integrals. Finally, the systematic approach taken by the author to bound integrals formally demonstrates that a certain degree of systematisation is possible, and the author's growing understanding of metaprogramming has led to the birth---though not much more---of ideas that would make formal integral calculations more user-friendly.

The road ahead may be uncertain, but it is safe to say it is filled with potential. In conclusion, the end of this M4R is only the beginning.