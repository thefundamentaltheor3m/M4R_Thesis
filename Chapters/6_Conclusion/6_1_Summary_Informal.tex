\section{Viazovska's Monumental Breakthrough}

It is with good reason that Cohn writes,
\begin{quote}
    Overall, this proof feels like a miracle.Everything falls beautifully into place, with Viazovska’s constructions having just enough flexibility to complete the proof in a unique way. \ldots Viazovska is a master of special functions, whose work would surely have excited Jacobi and Ramanujan
\end{quote}
\cite[p.21]{CohnOnViazovskaICM}. Viazovska's ingenuity stems from the manner in which she overcame the difficulties in controlling funtions and their Fourier transforms simultaneously: realising that the magic function would need to be an integral transform and that the function being transformed would need to admit numerous change of variable properties, she tapped into the theory of modular forms to find precisely such a function.

Over the course of this thesis, we have explored the specifics of how she does this. When we first encountered modular forms in \Cref{Ch2:Sec:ModForms}, it was already remarkable to observe that these 