\section{Viazovska's Monumental Breakthrough}

It is with good reason that Cohn writes,
\begin{quote}
    Overall, this proof feels like a miracle.Everything falls beautifully into place, with Viazovska’s constructions having just enough flexibility to complete the proof in a unique way. \ldots Viazovska is a master of special functions, whose work would surely have excited Jacobi and Ramanujan
\end{quote}
\cite[p.21]{CohnOnViazovskaICM}. Viazovska's ingenuity stems from the manner in which she overcame the difficulties in controlling funtions and their Fourier transforms simultaneously: realising that the magic function would need to be an integral transform and that the function being transformed would need to admit numerous change of variable properties, she tapped into the theory of modular forms to find precisely such a function.

Over the course of this thesis, we have explored the specifics of the solution. The first important insight were the linear programming bounds on sphere packing densities, which reduced a famously difficult geometric problem into a deceptively difficult analytic problem. The second was that there exist a remarkable class of functions, encountered in \Cref{Ch2:Sec:ModForms}, that obey numerous transformation rules that make them perfect candidates for navigating the symmetries of Möbius changes of variables. We began to get an inkling of their relevance when we observed, in \Cref{Ch3:Chapter}, that the problem of finding a magic function for the Cohn-Elkies bounds could be divided into two problems: finding $\pm 1$-Fourier eigenfunctions with double zeroes at lattice points and expressing them as an appropriate linear combination. We further established the relevance of the theory of modular forms by conceiving that their numerous change of variable properties might make them good candidates for integral transforms to construct the eigenfunctions. In \Cref{Ch4:Chapter}, we saw that Viazovska did something similar: she combined functions that were either modular forms or closely enough related to obey similar properties to form Schwartz functions that were well-behaved under precisely those variable changes needed to relate them to their Fourier transforms. We discussed her construction in tremendous detail, identifying previously formalised results that can be used in our formalisation as well as gaps in \mathlib\ that would need to be filled to successfully formalise Viazovska's result. Finally, in \Cref{Ch5:Chapter}, we gave an overview of techniques developed thus far in the formalisation to overcome some, though admittedly not all, of the difficulties identified in \Cref{Ch4:Chapter}.