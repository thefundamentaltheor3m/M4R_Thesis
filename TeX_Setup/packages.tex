%% Language and font encodings
\usepackage[english]{babel}
\usepackage[utf8]{inputenc}
\usepackage[T1]{fontenc}

%% Imperial Recommended Packages
\usepackage{afterpage}

% Lean colour configurations
\usepackage{color}
\definecolor{keywordcolor}{rgb}{0.7, 0.1, 0.1}   % red
\definecolor{tacticcolor}{rgb}{0.0, 0.1, 0.6}    % blue
\definecolor{commentcolor}{rgb}{0.4, 0.4, 0.4}   % grey
\definecolor{symbolcolor}{rgb}{0.0, 0.1, 0.6}    % blue
\definecolor{sortcolor}{rgb}{0.1, 0.5, 0.1}      % green
\definecolor{attributecolor}{rgb}{0.7, 0.1, 0.1} % red
\definecolor{backcolour}{rgb}{0.95,0.95,0.92}

% Listings (for displaying code):
\usepackage{listings}
\def\lstlanguagefiles{TeX_Setup/lstlean.tex}
\lstset{
    % frame = single, 
    % framexleftmargin=15pt,
    language = lean,
    numbers = left,
    backgroundcolor=\color{backcolour},
    inputencoding=ansinew,
    % extendchars=\true
}

% ----------- Algorithm2e setup
\usepackage[ruled,vlined]{algorithm2e}
\makeatletter
\renewcommand{\SetKwInOut}[2]{%
  \sbox\algocf@inoutbox{\KwSty{#2}\algocf@typo:}%
  \expandafter\ifx\csname InOutSizeDefined\endcsname\relax% if first time used
    \newcommand\InOutSizeDefined{}\setlength{\inoutsize}{\wd\algocf@inoutbox}%
    \sbox\algocf@inoutbox{\parbox[t]{\inoutsize}{\KwSty{#2}\algocf@typo:\hfill}~}\setlength{\inoutindent}{\wd\algocf@inoutbox}%
  \else% else keep the larger dimension
    \ifdim\wd\algocf@inoutbox>\inoutsize%
    \setlength{\inoutsize}{\wd\algocf@inoutbox}%
    \sbox\algocf@inoutbox{\parbox[t]{\inoutsize}{\KwSty{#2}\algocf@typo:\hfill}~}\setlength{\inoutindent}{\wd\algocf@inoutbox}%
    \fi%
  \fi% the dimension of the box is now defined.
  \algocf@newcommand{#1}[1]{%
    \ifthenelse{\boolean{algocf@inoutnumbered}}{\relax}{\everypar={\relax}}%
%     {\let\\\algocf@newinout\hangindent=\wd\algocf@inoutbox\hangafter=1\parbox[t]{\inoutsize}{\KwSty{#2}\algocf@typo\hfill:}~##1\par}%
    {\let\\\algocf@newinout\hangindent=\inoutindent\hangafter=1\parbox[t]{\inoutsize}{\KwSty{#2}\algocf@typo:\hfill}~##1\par}%
    \algocf@linesnumbered% reset the numbering of the lines
  }}%
\makeatother
% --------- end algorithm2e setup

\usepackage{bm}
\usepackage[normalem]{ulem}

\usepackage[colorinlistoftodos]{todonotes}

% I have all of their other recommended packages somewhere on here.

\usepackage[most]{tcolorbox}
\usepackage{authblk}  % Lets you add an \affil{} to your title, stating your affiliation {eg. Sigma Mathematics Society}
\usepackage{ragged2e}
\usepackage{csquotes}
\usepackage{pdfpages}



\usepackage{xfrac}
\usepackage{cancel}

\usepackage[inline]{enumitem}

%\usepackage{tgpagella}

\usepackage{blindtext}
\usepackage{lipsum}
\usepackage{verbatim}
\usepackage{hyperref}
\hypersetup{
    citebordercolor = 1 1 1,
    linkbordercolor = 1 1 1,
    filebordercolor = 1 1 1,
    menubordercolor = 1 1 1,
    urlbordercolor = 1 1 1,
    colorlinks  =   true,
    linkcolor   =   blue,
    citecolor   =   magenta,
    urlcolor    =   blue
}

% This project uses natbib instead. See end of format file.
% \usepackage{biblatex} % Modify citation format using [style=yourstyle] parameter--eg \usepackage[style=mla-new]{biblatex}
% \bibliography{TeX_Setup/References.bib}
% \addbibresource{TeX_Setup/References.bib}

\usepackage{cancel}
\usepackage{amssymb}
\usepackage{amsmath}
% \usepackage{amsthm}  % In `environments.tex`
%\usepackage{MnSymbol}
\usepackage{mathrsfs}
\usepackage{mathtools}
% \usepackage{mathabx}
\usepackage{mathdots}
\usepackage{yhmath}

\usepackage{array}
\usepackage{booktabs}
\usepackage{longtable}

\usepackage{graphicx}
\newcommand\sbullet[1][.5]{\mathbin{\vcenter{\hbox{\scalebox{#1}{$\bullet$}}}}}  % Bullet of customisable size
\usepackage{wrapfig}
% Imperial-recommended `caption` setup
\usepackage{caption}
\captionsetup[figure]{labelfont={bf}, name={Figure}, justification=centering}
\captionsetup[table]{labelfont={bf}, name={Table}}
\usepackage{subcaption}
\usepackage{tikz}
\usepackage{float}

% Tikz
\usepackage{tikz-cd}
\usepackage{tikz-3dplot}
\usetikzlibrary{positioning}
\usetikzlibrary{cd}
\usetikzlibrary{shapes.geometric}
\usepackage{pgfplots}
\usepackage{mathrsfs}
\usetikzlibrary{arrows}
\usepackage{qtree}