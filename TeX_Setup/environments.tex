\usepackage{amsthm}
% Gives theorem and definition names the same font style as the words "theorem"/"definition": see https://tex.stackexchange.com/questions/43966/how-to-make-the-optional-title-of-a-theorem-bold-with-amsthm
\makeatletter
\def\th@plain{%
  \thm@notefont{}% same as heading font
  \itshape % body font
}
\def\th@definition{%
  \thm@notefont{}% same as heading font
  \normalfont % body font
}
\makeatother

\usepackage{cleveref}

\newtheorem*{theorem*}{Theorem}
\newtheorem{theorem}{Theorem}[section]
\newtheorem{corollary}[theorem]{Corollary}%[theorem]
\newtheorem{lemma}[theorem]{Lemma}
\newtheorem{claim}[theorem]{Claim}
\newtheorem{conjecture}[theorem]{Conjecture}
% \newtheorem{algorithm}[theorem]{Algorithm}  % Defined in algorithm2e
\newtheorem{proposition}[theorem]{Proposition}

\newtheorem{problem}[theorem]{Problem}
\newenvironment{boxproblem}{
    \begin{tcolorbox}[colback=yellow!15!white,colframe=orange, breakable, enhanced]\begin{problem}
}{
    \end{problem}\end{tcolorbox}
}

\newenvironment{boxtheorem}{
    \begin{tcolorbox}[colback=yellow!15!white,colframe=orange, breakable, enhanced]\begin{theorem}
}{
    \end{theorem}\end{tcolorbox}
}
\newenvironment{boxproposition}{
    \begin{tcolorbox}[colback=yellow!15!white,colframe=orange, breakable, enhanced]\begin{proposition}
}{
    \end{proposition}\end{tcolorbox}
}
\newenvironment{boxlemma}{
    \begin{tcolorbox}[colback=yellow!15!white,colframe=orange, breakable, enhanced]\begin{lemma}
}{
    \end{lemma}\end{tcolorbox}
}
\newenvironment{boxcorollary}{
    \begin{tcolorbox}[colback=yellow!15!white,colframe=orange, breakable, enhanced]\begin{corollary}
}{
    \end{corollary}\end{tcolorbox}
}
\newenvironment{boxconjecture}{
    \begin{tcolorbox}[colback=yellow!15!white,colframe=orange, breakable, enhanced]\begin{conjecture}
}{
    \end{conjecture}\end{tcolorbox}
}


\theoremstyle{remark}
\newtheorem*{remark}{Remark}
\newtheorem*{solution}{Solution}

\theoremstyle{definition}
\newtheorem{definition}[theorem]{Definition}
\newenvironment{boxdefinition}{
    \begin{tcolorbox}[colback=cyan!10!white,colframe=cyan!70!black, breakable, enhanced]\begin{definition}
}{
    \end{definition}\end{tcolorbox}
}
\newtheorem*{convention}{Convention}
\newenvironment{boxconvention}{
    \begin{tcolorbox}[colback=magenta!3!white,colframe=magenta!70!black, breakable, enhanced]\begin{convention}
}{
    \end{convention}\end{tcolorbox}
}
\newtheorem*{notation}{Notation}
\newenvironment{boxnotation}{
    \begin{tcolorbox}[colback=magenta!3!white,colframe=magenta!70!black, breakable, enhanced]\begin{notation}
}{
    \end{notation}\end{tcolorbox}
}
\newtheorem{example}[theorem]{Example}
\newenvironment{boxexample}{
    \begin{tcolorbox}[colframe=green!30!black, breakable, enhanced, breakable, enhanced]\begin{example}
}{
    \end{example}\end{tcolorbox}
}
\newtheorem{nexample}[theorem]{Non-Example}
\newenvironment{boxnexample}{
    \begin{tcolorbox}[colframe=red!50!black, breakable, enhanced]\begin{nexample}
}{
    \end{nexample}\end{tcolorbox}
}
\newtheorem{cexample}[theorem]{Counterexample}
\newenvironment{boxcexample}{
    \begin{tcolorbox}[colframe=red!50!black, breakable, enhanced]\begin{cexample}
}{
    \end{cexample}\end{tcolorbox}
}

% \def\SMALLCOLWIDTH{1.5cm}
\newcolumntype{C}[1]{>{\centering\let\newline\\\arraybackslash\hspace{0pt}}m{#1}}

% Centering captions
% \renewcommand{\caption}[1]{\caption{\centering #1}}