%%%%%%%%%%%%%%%%%%%%%%%%%%%%%%%%%%%%%%%%%%%%%%%%%%%%%%%%%%%%%%%%%%%%%%%%%%%%%%%%%%%%%%%%%%%%%%%%%%%%%%%%%%%%
% CUSTOM COMMANDS
%%%%%%%%%%%%%%%%%%%%%%%%%%%%%%%%%%%%%%%%%%%%%%%%%%%%%%%%%%%%%%%%%%%%%%%%%%%%%%%%%%%%%%%%%%%%%%%%%%%%%%%%%%%%

% `sorry`

\newcommand{\sorry}{\textcolor{red}{\texttt{sorry}}}

% TIKZ:

\newcommand{\drawplane}{  % For use with TikZ
    \draw[step=0.5cm,gray,very thin] (-2.5,-2.5) grid (2.5,2.5);
    \draw[thick,->] (-2.5,0) -- (2.5,0); % node[anchor=west] {$x$};
    \draw[thick,->] (0,-2.5) -- (0,2.5); % node[anchor=south] {$y$};
    % \node[black, anchor=north east] at (0, 0) {$0$};
}
\newcommand{\latticecircle}[2]{
    \draw[fill=yellow, opacity=0.5] (#1, #2) circle (0.5);
    \node at (#1, #2) {\color{gray}{$\bullet$}};
}
\newcommand{\latticecirclegrey}[2]{
    \draw[fill=gray, opacity=0.75] (#1, #2) circle (0.5);
    \node at (#1, #2) {\color{black}{$\bullet$}};
}
\usetikzlibrary{decorations.markings}
\tikzset{->-/.style={decoration={
      markings,
      mark=at position #1 with {\arrow{>}}},postaction={decorate}
}}

\newenvironment{cd}{
    \begin{equation} \begin{tikzcd}
}{
    \end{tikzcd} \end{equation}
}
\newenvironment{cd*}{
    \begin{equation*} \begin{tikzcd}
}{
    \end{tikzcd} \end{equation*}
}

\newcommand{\drawsquare}[1]{
    \draw[thick, blue]
    (-{#1},-{#1}) node [anchor=north east] {C} --
    (-{#1}, {#1}) node[anchor=south east] {B} --
    ({#1}, {#1}) node[anchor=south west] {A} --
    ({#1}, -{#1}) node [anchor=north west] {D} -- cycle;
}

% DELIMITERS:

\newcommand{\parenth}[1]{\left( #1 \right)}
\newcommand{\brac}[1]{\left[ #1 \right]}
\newcommand{\set}[1]{\left\{ #1 \right\}}
\newcommand{\setst}[2]{\set{#1 \; \middle\vert \; #2}}
\newcommand{\abs}[1]{\left\lvert #1 \right\rvert}
\newcommand{\norm}[1]{\left\lVert #1 \right\rVert}
\newcommand{\floor}[1]{\left\lfloor #1 \right\rfloor}
\newcommand{\ceil}[1]{\left\lceil #1 \right\rceil}
\newcommand{\cycl}[1]{\left\langle #1 \right\rangle}
\newcommand{\grpres}[2]{\cycl{#1 \; \middle| \; #2}}  % \grpres{gens}{rels}

% FUNCTIONS:

\newcommand{\fx}{f\!\parenth{x}}
\newcommand{\fof}[1]{f\!\parenth{#1}}

\newcommand{\px}{p\!\parenth{x}}
\newcommand{\pof}[1]{p\!\parenth{#1}}
\newcommand{\pofbig}[1]{p\Big(#1\Big)}
\newcommand{\gx}{g\!\parenth{x}}
\newcommand{\gof}[1]{g\!\parenth{#1}}
\newcommand{\hx}{h\!\parenth{x}}
\newcommand{\hof}[1]{h\!\parenth{#1}}
\newcommand{\Tv}{T\!\parenth{v}}
\newcommand{\Tof}[1]{T\!\parenth{#1}}
\newcommand{\Tbar}{\overline{T}}
\newcommand{\Tbarof}[1]{\Tbar\!\parenth{#1}}
\newcommand{\Tbarv}{\Tbarof{v}}
\newcommand{\Sof}[1]{S\!\parenth{#1}}

\newcommand{\psin}[1]{\sin\!{\parenth{#1}}}
\newcommand{\pcos}[1]{\cos\!{\parenth{#1}}}
\newcommand{\ptan}[1]{\tan\!{\parenth{#1}}}
\newcommand{\sinsq}[1]{\sin^2\!\parenth{#1}}
\newcommand{\cossq}[1]{\cos^2\!\parenth{#1}}
\newcommand{\tansq}[1]{\tan^2\!\parenth{#1}}
\newcommand{\parcsin}[1]{\arcsin\!{\parenth{#1}}}
\newcommand{\parccos}[1]{\arccos\!{\parenth{#1}}}
\newcommand{\parctan}[1]{\arctan\!{\parenth{#1}}}

\newcommand{\logbase}[2]{\log_{#1}\!\parenth{#2}}
\newcommand{\nthroot}[2]{\sqrt[\leftroot{-3}\uproot{3} #1]{#2}}
\newcommand{\cbrt}[1]{\nthroot{3}{#1}}

\newcommand{\pgcd}[2]{\gcd\!\parenth{{#1},{#2}}}
\newcommand{\plcm}[2]{\operatorname{lcm}\!\parenth{{#1},{#2}}}
\newcommand{\pgcds}[1]{\gcd\!\parenth{#1}}
\newcommand{\plcms}[1]{\operatorname{lcm}\!\parenth{#1}}

\newcommand{\varphiof}[1]{\varphi\!\parenth{#1}}
\newcommand{\phiof}[1]{\phi\!\parenth{#1}}
\newcommand{\xiof}[1]{\xi\!\parenth{#1}}
\newcommand{\rhoof}[1]{\rho\!\parenth{#1}}
\newcommand{\pexp}[1]{\exp\!\parenth{#1}}

\newcommand{\inj}{\hookrightarrow}
\newcommand{\surj}{\twoheadrightarrow}

\DeclareMathOperator*{\argmin}{\arg\!\min}
\DeclareMathOperator*{\argmax}{\arg\!\max}

\newcommand{\of}[1]{\!\parenth{#1}}

% CALCULUS:

\newcommand{\dx}{\operatorname{d}\!x}
\newcommand{\dy}{\operatorname{d}\!y}
\newcommand{\diff}[1]{\operatorname{d}\!{#1}}
\newcommand{\dydx}{\frac{\dy}{\dx}}

% LINEAR ALGEBRA:

\newcommand{\mat}[3]{\operatorname{M}_{{#1} \times {#2}}\!\parenth{{#3}}}
\newcommand{\matsq}[2]{\operatorname{M}_{{#1} \times {#1}}\!\parenth{\mathbb{#2}}}
\newcommand{\MnR}{\operatorname{M}_{n}\!\parenth{\real}}
\newcommand{\MnC}{\operatorname{M}_{n}\!\parenth{\C}}
\newcommand{\Mn}[2]{\operatorname{M}_{#1}\!\parenth{#2}}
\newcommand{\GL}[1]{\operatorname{GL}\!\parenth{#1}}
\newcommand{\SL}[1]{\operatorname{SL}\!\parenth{#1}}
\newcommand{\PGL}[1]{\operatorname{PGL}\!\parenth{#1}}
\newcommand{\PSL}[1]{\operatorname{PSL}\!\parenth{#1}}

\newcommand{\Span}[1]{\operatorname{Span}\!\parenth{#1}}

\newcommand{\pdim}[1]{\dim\!\parenth{#1}}

\newcommand{\RSp}[1]{\operatorname{RSp}\!\parenth{#1}}
\newcommand{\CSp}[1]{\operatorname{CSp}\!\parenth{#1}}
\newcommand{\rank}[1]{\operatorname{rank}\!\parenth{#1}}
\newcommand{\pim}[1]{\operatorname{im}\!\parenth{#1}}
\newcommand{\pker}[1]{\operatorname{ker}\!\parenth{#1}}
\newcommand{\pdet}[1]{\det\!\parenth{#1}}

\newcommand{\cA}[1]{c_A \! \parenth{#1}}
\newcommand{\cB}[1]{c_B \! \parenth{#1}}
\newcommand{\mA}[1]{m_A \! \parenth{#1}}
\newcommand{\mB}[1]{m_B \! \parenth{#1}}
\newcommand{\cAx}{\cA{x}}
\newcommand{\cBx}{\cB{x}}
\newcommand{\mAx}{\mA{x}}
\newcommand{\mBx}{\mB{x}}
\newcommand{\cof}[2]{c_{#1}\!\parenth{#2}}
\newcommand{\mof}[2]{m_{#1}\!\parenth{#2}}

\newcommand{\Cof}[1]{C\!\parenth{#1}}
\newcommand{\+}{\oplus}
\newcommand{\pdeg}[1]{\deg\!\parenth{#1}}

\newcommand{\RCF}[1]{\operatorname{RCF}\!\parenth{#1}}
\newcommand{\JCF}[1]{\operatorname{JCF}\!\parenth{#1}}

\newcommand{\Jsub}[2]{J_{#1}\!\parenth{#2}}

\newcommand{\Aof}[1]{A\!\parenth{#1}}

\newcommand{\Tsub}[2]{T_{#1}\!\parenth{#2}}

\newcommand{\Tr}[1]{\operatorname{Tr}\!\parenth{#1}}

\newcommand{\diag}[1]{\operatorname{diag}\!\parenth{#1}}

% Measure Theory

\newcommand{\muof}[1]{\mu\!\parenth{#1}}
\newcommand{\muA}{\muof{A}}
\newcommand{\muB}{\muof{B}}
\newcommand{\mutof}[1]{\Tilde{\mu}\!\parenth{#1}}
\newcommand{\mustof}[1]{\mu^*\!\parenth{#1}}
\newcommand{\calF}{\mathcal{F}}
\newcommand{\calA}{\mathcal{A}}
\newcommand{\calB}{\mathcal{B}}
\newcommand{\calC}{\mathcal{C}}
\newcommand{\sigmaof}[1]{\sigma\!\parenth{#1}}
\newcommand{\psiof}[1]{\psi\!\parenth{#1}}

% Intervals

\newcommand{\oc}[2]{\left(#1, #2\right]}
\newcommand{\co}[2]{\left[#1, #2\right)}

% GROUPS/ALGEBRA IN GENERAL:

\newcommand{\inv}{^{-1}}
\newcommand{\Sym}[1]{\operatorname{Sym}\!\parenth{#1}}
\newcommand{\ord}[1]{\operatorname{ord}\!\parenth{#1}}
\newcommand{\supp}[1]{\operatorname{supp}\!\parenth{#1}}
\newcommand{\sgn}[1]{\operatorname{sgn}\!\parenth{#1}}
\newcommand{\tcyc}[2]{\begin{pmatrix} #1 & #2 \end{pmatrix}}

\newcommand{\nsg}{\trianglelefteq}

\newcommand{\Rmul}{R^{\times}}
\newcommand{\kmul}{k^{\times}}
\newcommand{\kX}{k\!\brac{X}}
\newcommand{\RX}{R\!\brac{X}}

\newcommand{\quotient}[2]{
    \left.\raisebox{.2em}{${#1}$} \middle/ \raisebox{-.2em}{${#2}$} \right.
}

\newcommand{\Vof}[1]{V\!\parenth{#1}}

\newcommand{\Frac}[1]{\operatorname{Frac}\!\parenth{#1}}

\newcommand{\Spec}[1]{\operatorname{Spec}\!\parenth{#1}}

\newcommand{\id}{\operatorname{id}}

\newcommand{\pchar}[1]{\operatorname{char}\!\parenth{#1}}

\newcommand{\Hom}{\operatorname{Hom}}

\newcommand{\Zof}[1]{\operatorname{Z}\!\parenth{#1}}

\newcommand{\Aut}[1]{\operatorname{Aut}\!\parenth{#1}}

% NUMBER SETS:

\newcommand{\real}{\mathbb{R}}
\newcommand{\rational}{\mathbb{Q}}
\newcommand{\naturalnum}{\mathbb{N}}
\newcommand{\integers}{\mathbb{Z}}
\newcommand{\complex}{\mathbb{C}}

\newcommand{\Field}{\mathbb{F}}
\newcommand{\Q}{\mathbb{Q}}
\newcommand{\R}{\real}
\newcommand{\N}{\naturalnum}
\newcommand{\Z}{\integers}
\newcommand{\C}{\complex}
\newcommand{\B}{\mathcal{B}}
\newcommand{\I}{\mathcal{I}}
\newcommand{\J}{\mathcal{J}}

\newcommand{\psup}[1]{\sup\!\parenth{#1}}
\newcommand{\pinf}[1]{\inf\!\parenth{#1}}

% LOGIC:

\newcommand{\ergo}{\therefore}
\newcommand{\bcos}{\because}
\newcommand{\st}{\text{ s.t. }}


% -------------------- SPECIFIC TO PROJECT -------------------- %

% TERMINOLOGY

\newcommand{\CELP}{Cohn-Elkies Linear Programming Bound}
\newcommand{\CEC}{Cohn-Elkies Conditions}

% NOTATION

\newcommand{\Pa}{\mathcal{P}}
\newcommand{\Vol}{\operatorname{Vol}}
\newcommand{\Volof}[1]{\Vol\!\parenth{#1}}
\newcommand{\Sch}{\mathcal{S}}
\renewcommand{\hat}{\widehat}
\newcommand{\F}{\mathcal{F}}
\newcommand{\periodic}{\operatorname{periodic}}

\newcommand{\Halfplane}{\mathbb{H}} % Using Lean notation for the upper-half plane instead of Viazovska's

\newcommand{\rad}{_{\operatorname{rad}}}

% LEAN

\newcommand{\mathlib}{\texttt{mathlib}}
